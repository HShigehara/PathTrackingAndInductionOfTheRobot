<!DOCTYPE html PUBLIC "-//W3C//DTD XHTML 1.0 Transitional//EN" "http://www.w3.org/TR/xhtml1/DTD/xhtml1-transitional.dtd">
<html xmlns="http://www.w3.org/1999/xhtml">
<head>
<meta http-equiv="Content-Type" content="text/xhtml;charset=UTF-8"/>
<meta http-equiv="X-UA-Compatible" content="IE=9"/>
<meta name="generator" content="Doxygen 1.8.6"/>
<title>PathTrackingAndInductionOfTheRobot: main.cpp ソースファイル</title>
<link href="tabs.css" rel="stylesheet" type="text/css"/>
<script type="text/javascript" src="jquery.js"></script>
<script type="text/javascript" src="dynsections.js"></script>
<link href="navtree.css" rel="stylesheet" type="text/css"/>
<script type="text/javascript" src="resize.js"></script>
<script type="text/javascript" src="navtree.js"></script>
<script type="text/javascript">
  $(document).ready(initResizable);
  $(window).load(resizeHeight);
</script>
<link href="search/search.css" rel="stylesheet" type="text/css"/>
<script type="text/javascript" src="search/search.js"></script>
<script type="text/javascript">
  $(document).ready(function() { searchBox.OnSelectItem(0); });
</script>
<link href="doxygen.css" rel="stylesheet" type="text/css" />
</head>
<body>
<div id="top"><!-- do not remove this div, it is closed by doxygen! -->
<div id="titlearea">
<table cellspacing="0" cellpadding="0">
 <tbody>
 <tr style="height: 56px;">
  <td style="padding-left: 0.5em;">
   <div id="projectname">PathTrackingAndInductionOfTheRobot
   </div>
  </td>
 </tr>
 </tbody>
</table>
</div>
<!-- end header part -->
<!-- 構築: Doxygen 1.8.6 -->
<script type="text/javascript">
var searchBox = new SearchBox("searchBox", "search",false,'検索');
</script>
  <div id="navrow1" class="tabs">
    <ul class="tablist">
      <li><a href="index.html"><span>総合概要</span></a></li>
      <li><a href="annotated.html"><span>クラス</span></a></li>
      <li class="current"><a href="files.html"><span>ファイル</span></a></li>
      <li>
        <div id="MSearchBox" class="MSearchBoxInactive">
        <span class="left">
          <img id="MSearchSelect" src="search/mag_sel.png"
               onmouseover="return searchBox.OnSearchSelectShow()"
               onmouseout="return searchBox.OnSearchSelectHide()"
               alt=""/>
          <input type="text" id="MSearchField" value="検索" accesskey="S"
               onfocus="searchBox.OnSearchFieldFocus(true)" 
               onblur="searchBox.OnSearchFieldFocus(false)" 
               onkeyup="searchBox.OnSearchFieldChange(event)"/>
          </span><span class="right">
            <a id="MSearchClose" href="javascript:searchBox.CloseResultsWindow()"><img id="MSearchCloseImg" border="0" src="search/close.png" alt=""/></a>
          </span>
        </div>
      </li>
    </ul>
  </div>
  <div id="navrow2" class="tabs2">
    <ul class="tablist">
      <li><a href="files.html"><span>ファイル一覧</span></a></li>
      <li><a href="globals.html"><span>ファイルメンバ</span></a></li>
    </ul>
  </div>
</div><!-- top -->
<div id="side-nav" class="ui-resizable side-nav-resizable">
  <div id="nav-tree">
    <div id="nav-tree-contents">
      <div id="nav-sync" class="sync"></div>
    </div>
  </div>
  <div id="splitbar" style="-moz-user-select:none;" 
       class="ui-resizable-handle">
  </div>
</div>
<script type="text/javascript">
$(document).ready(function(){initNavTree('main_8cpp_source.html','');});
</script>
<div id="doc-content">
<!-- window showing the filter options -->
<div id="MSearchSelectWindow"
     onmouseover="return searchBox.OnSearchSelectShow()"
     onmouseout="return searchBox.OnSearchSelectHide()"
     onkeydown="return searchBox.OnSearchSelectKey(event)">
<a class="SelectItem" href="javascript:void(0)" onclick="searchBox.OnSelectItem(0)"><span class="SelectionMark">&#160;</span>全て</a><a class="SelectItem" href="javascript:void(0)" onclick="searchBox.OnSelectItem(1)"><span class="SelectionMark">&#160;</span>クラス</a><a class="SelectItem" href="javascript:void(0)" onclick="searchBox.OnSelectItem(2)"><span class="SelectionMark">&#160;</span>ファイル</a><a class="SelectItem" href="javascript:void(0)" onclick="searchBox.OnSelectItem(3)"><span class="SelectionMark">&#160;</span>関数</a><a class="SelectItem" href="javascript:void(0)" onclick="searchBox.OnSelectItem(4)"><span class="SelectionMark">&#160;</span>変数</a><a class="SelectItem" href="javascript:void(0)" onclick="searchBox.OnSelectItem(5)"><span class="SelectionMark">&#160;</span>型定義</a><a class="SelectItem" href="javascript:void(0)" onclick="searchBox.OnSelectItem(6)"><span class="SelectionMark">&#160;</span>マクロ定義</a></div>

<!-- iframe showing the search results (closed by default) -->
<div id="MSearchResultsWindow">
<iframe src="javascript:void(0)" frameborder="0" 
        name="MSearchResults" id="MSearchResults">
</iframe>
</div>

<div class="header">
  <div class="headertitle">
<div class="title">main.cpp</div>  </div>
</div><!--header-->
<div class="contents">
<a href="main_8cpp.html">[詳解]</a><div class="fragment"><div class="line"><a name="l00001"></a><span class="lineno">    1</span>&#160;<span class="comment">/*</span></div>
<div class="line"><a name="l00002"></a><span class="lineno">    2</span>&#160;<span class="comment"> * @file main.cpp</span></div>
<div class="line"><a name="l00003"></a><span class="lineno">    3</span>&#160;<span class="comment"> * @brief main関数</span></div>
<div class="line"><a name="l00004"></a><span class="lineno">    4</span>&#160;<span class="comment"> * @date 2014.10.15</span></div>
<div class="line"><a name="l00005"></a><span class="lineno">    5</span>&#160;<span class="comment"> * @author H.Shigehara</span></div>
<div class="line"><a name="l00006"></a><span class="lineno">    6</span>&#160;<span class="comment"> */</span></div>
<div class="line"><a name="l00007"></a><span class="lineno">    7</span>&#160;</div>
<div class="line"><a name="l00008"></a><span class="lineno">    8</span>&#160;<span class="comment">/* ヘッダファイルのインクルード */</span></div>
<div class="line"><a name="l00009"></a><span class="lineno">    9</span>&#160;<span class="preprocessor">#include &quot;<a class="code" href="_path_tracking_and_induction_of_the_robot_8hpp.html">PathTrackingAndInductionOfTheRobot.hpp</a>&quot;</span></div>
<div class="line"><a name="l00010"></a><span class="lineno">   10</span>&#160;<span class="preprocessor">#include &quot;<a class="code" href="_kinect_8hpp.html">Kinect.hpp</a>&quot;</span></div>
<div class="line"><a name="l00011"></a><span class="lineno">   11</span>&#160;<span class="preprocessor">#include &quot;<a class="code" href="_image_processing_8hpp.html">ImageProcessing.hpp</a>&quot;</span></div>
<div class="line"><a name="l00012"></a><span class="lineno">   12</span>&#160;<span class="preprocessor">#include &quot;<a class="code" href="_drawing_8hpp.html">Drawing.hpp</a>&quot;</span></div>
<div class="line"><a name="l00013"></a><span class="lineno">   13</span>&#160;<span class="preprocessor">#include &quot;<a class="code" href="_system_8hpp.html">System.hpp</a>&quot;</span></div>
<div class="line"><a name="l00014"></a><span class="lineno">   14</span>&#160;<span class="preprocessor">#include &quot;<a class="code" href="_least_square_method_8hpp.html">LeastSquareMethod.hpp</a>&quot;</span> <span class="comment">//(c49)</span></div>
<div class="line"><a name="l00015"></a><span class="lineno">   15</span>&#160;<span class="preprocessor">#include &quot;<a class="code" href="_point_cloud_library_8hpp.html">PointCloudLibrary.hpp</a>&quot;</span> <span class="comment">//ポイントクラウド処理のヘッダを追加(c57)</span></div>
<div class="line"><a name="l00016"></a><span class="lineno">   16</span>&#160;<span class="preprocessor">#include &quot;<a class="code" href="_e_v3_control_8hpp.html">EV3Control.hpp</a>&quot;</span> <span class="comment">//EV3を制御するようのクラスを作成(c80)</span></div>
<div class="line"><a name="l00017"></a><span class="lineno">   17</span>&#160;</div>
<div class="line"><a name="l00018"></a><span class="lineno">   18</span>&#160;<span class="comment">/* グローバル変数 */</span></div>
<div class="line"><a name="l00019"></a><span class="lineno">   19</span>&#160;<span class="comment">//画像データ</span></div>
<div class="line"><a name="l00020"></a><span class="lineno"><a class="line" href="_path_tracking_and_induction_of_the_robot_8hpp.html#aabb27b8973575043030df51be47cd24a">   20</a></span>&#160;Mat <a class="code" href="main_8cpp.html#aabb27b8973575043030df51be47cd24a">image</a>; </div>
<div class="line"><a name="l00021"></a><span class="lineno">   21</span>&#160;</div>
<div class="line"><a name="l00022"></a><span class="lineno"><a class="line" href="_path_tracking_and_induction_of_the_robot_8hpp.html#abefb498e9a643f68bb3d37c22953ddad">   22</a></span>&#160;<span class="keywordtype">char</span> <a class="code" href="main_8cpp.html#abefb498e9a643f68bb3d37c22953ddad">directoryName</a>[<a class="code" href="stdafx_8h.html#a5d8ba032e2fdf7b2561f508164124f3e">NOC</a>]; </div>
<div class="line"><a name="l00023"></a><span class="lineno">   23</span>&#160;</div>
<div class="line"><a name="l00024"></a><span class="lineno">   24</span>&#160;<span class="comment">//(c25)</span></div>
<div class="line"><a name="l00025"></a><span class="lineno"><a class="line" href="_path_tracking_and_induction_of_the_robot_8hpp.html#a71a276a0dc4ef6aa35740b58f10bcb39">   25</a></span>&#160;<span class="keywordtype">bool</span> <a class="code" href="main_8cpp.html#a71a276a0dc4ef6aa35740b58f10bcb39">selectObject</a> = <span class="keyword">false</span>; </div>
<div class="line"><a name="l00026"></a><span class="lineno"><a class="line" href="_path_tracking_and_induction_of_the_robot_8hpp.html#a88f176e4e65e22a17093a8cd24003a66">   26</a></span>&#160;<span class="keywordtype">int</span> <a class="code" href="main_8cpp.html#a88f176e4e65e22a17093a8cd24003a66">trackObject</a> = 0; </div>
<div class="line"><a name="l00027"></a><span class="lineno"><a class="line" href="_path_tracking_and_induction_of_the_robot_8hpp.html#a903d0d8820c696aaa1170c50deb3633f">   27</a></span>&#160;Point <a class="code" href="main_8cpp.html#a903d0d8820c696aaa1170c50deb3633f">origin</a>; </div>
<div class="line"><a name="l00028"></a><span class="lineno"><a class="line" href="_path_tracking_and_induction_of_the_robot_8hpp.html#a15633c47538e3790df1e008a7f6dbfea">   28</a></span>&#160;Rect <a class="code" href="main_8cpp.html#a15633c47538e3790df1e008a7f6dbfea">selection</a>; </div>
<div class="line"><a name="l00029"></a><span class="lineno">   29</span>&#160;<span class="keywordtype">void</span> <a class="code" href="main_8cpp.html#ab6ccc6c8d80970d4178c943533533860">onMouse</a>(<span class="keywordtype">int</span> event, <span class="keywordtype">int</span> x, <span class="keywordtype">int</span> y, <span class="keywordtype">int</span> flags, <span class="keywordtype">void</span>* param); </div>
<div class="line"><a name="l00030"></a><span class="lineno">   30</span>&#160;</div>
<div class="line"><a name="l00036"></a><span class="lineno"><a class="line" href="main_8cpp.html#ae66f6b31b5ad750f1fe042a706a4e3d4">   36</a></span>&#160;<span class="keywordtype">int</span> <a class="code" href="main_8cpp.html#ae66f6b31b5ad750f1fe042a706a4e3d4">main</a>()</div>
<div class="line"><a name="l00037"></a><span class="lineno">   37</span>&#160;{</div>
<div class="line"><a name="l00038"></a><span class="lineno">   38</span>&#160;    RETRY: <span class="comment">//goto文.再計測をやり直す場合</span></div>
<div class="line"><a name="l00039"></a><span class="lineno">   39</span>&#160;    <span class="comment">//インスタンスの生成</span></div>
<div class="line"><a name="l00040"></a><span class="lineno">   40</span>&#160;    <a class="code" href="class_kinect.html">Kinect</a> kinect; <span class="comment">//Kinectクラスのインスタンスを生成</span></div>
<div class="line"><a name="l00041"></a><span class="lineno">   41</span>&#160;    <a class="code" href="class_system.html">System</a> sys; <span class="comment">//システム的なメソッドをまとめているクラス</span></div>
<div class="line"><a name="l00042"></a><span class="lineno">   42</span>&#160;    <a class="code" href="class_image_processing.html">ImageProcessing</a> imgproc; <span class="comment">//Imageprocessingクラスのインスタンスを生成</span></div>
<div class="line"><a name="l00043"></a><span class="lineno">   43</span>&#160;    <a class="code" href="class_drawing.html">Drawing</a> draw; <span class="comment">//drawingクラスのインスタンスを生成</span></div>
<div class="line"><a name="l00044"></a><span class="lineno">   44</span>&#160;    <a class="code" href="class_least_square_method.html">LeastSquareMethod</a> lsm; <span class="comment">//最小二乗法を行うクラスのインスタンスを生成(c49)</span></div>
<div class="line"><a name="l00045"></a><span class="lineno">   45</span>&#160;    <a class="code" href="class_point_cloud_library.html">PointCloudLibrary</a> pointcloudlibrary(<span class="comment">/*false*/</span><span class="keyword">true</span>, <span class="comment">/*false*/</span><span class="keyword">true</span>, <span class="comment">/*false*/</span><span class="keyword">true</span>, <span class="keyword">false</span>, <span class="keyword">false</span><span class="comment">/*true*/</span>); <span class="comment">//PointCloudLibraryクラスのインスタンスを生成(c57)</span></div>
<div class="line"><a name="l00046"></a><span class="lineno">   46</span>&#160;    <a class="code" href="class_e_v3_control.html">EV3Control</a> ev3control; <span class="comment">//EV3を制御する用のクラスを作成(c80)</span></div>
<div class="line"><a name="l00047"></a><span class="lineno">   47</span>&#160;</div>
<div class="line"><a name="l00048"></a><span class="lineno">   48</span>&#160;    <span class="comment">//変数の宣言</span></div>
<div class="line"><a name="l00049"></a><span class="lineno">   49</span>&#160;    <span class="keywordtype">bool</span> saveev3route_flag = <span class="keyword">false</span>; <span class="comment">//EV3の軌道を保存するためのフラグ(c82)</span></div>
<div class="line"><a name="l00050"></a><span class="lineno">   50</span>&#160;    <span class="keywordtype">int</span> save_count = 0; <span class="comment">//pキーを入力するたびに何回保存したかを数える変数</span></div>
<div class="line"><a name="l00051"></a><span class="lineno">   51</span>&#160;</div>
<div class="line"><a name="l00052"></a><span class="lineno">   52</span>&#160;    <span class="comment">//画像関係の変数</span></div>
<div class="line"><a name="l00053"></a><span class="lineno">   53</span>&#160;    Mat flip_image; <span class="comment">//確認用の反転した画像</span></div>
<div class="line"><a name="l00054"></a><span class="lineno">   54</span>&#160;    Mat current_image; <span class="comment">//現在のフレームの画像(c75)</span></div>
<div class="line"><a name="l00055"></a><span class="lineno">   55</span>&#160;    Mat bin_image; <span class="comment">//背景差分によって得られた二値画像(c75)</span></div>
<div class="line"><a name="l00056"></a><span class="lineno">   56</span>&#160;    Mat background_image; <span class="comment">//背景画像(c75)</span></div>
<div class="line"><a name="l00057"></a><span class="lineno">   57</span>&#160;    Mat background_gray_image; <span class="comment">//背景画像を二値化した画像</span></div>
<div class="line"><a name="l00058"></a><span class="lineno">   58</span>&#160;</div>
<div class="line"><a name="l00059"></a><span class="lineno">   59</span>&#160;    <span class="comment">//ポイントクラウド関係の変数(c57)</span></div>
<div class="line"><a name="l00060"></a><span class="lineno">   60</span>&#160;    pcl::PointCloud&lt;pcl::PointXYZRGB&gt;::Ptr cloud; <span class="comment">//処理を受け取る点群</span></div>
<div class="line"><a name="l00061"></a><span class="lineno">   61</span>&#160;    <span class="comment">//pcl::PointCloud&lt;pcl::Normal&gt;::Ptr cloud_normals(new pcl::PointCloud&lt;pcl::Normal&gt;); //!&lt;法線を格納する変数(c84)</span></div>
<div class="line"><a name="l00062"></a><span class="lineno">   62</span>&#160;</div>
<div class="line"><a name="l00063"></a><span class="lineno">   63</span>&#160;    <span class="comment">//EV3ユニットの平面の係数(c78)</span></div>
<div class="line"><a name="l00064"></a><span class="lineno">   64</span>&#160;    </div>
<div class="line"><a name="l00065"></a><span class="lineno">   65</span>&#160;    <span class="comment">//メインの処理</span></div>
<div class="line"><a name="l00066"></a><span class="lineno">   66</span>&#160;    <span class="keywordflow">try</span>{</div>
<div class="line"><a name="l00067"></a><span class="lineno">   67</span>&#160;        sys.<a class="code" href="class_system.html#a15325f6b106b804af8432d102945535c">startMessage</a>(); <span class="comment">//プログラム開始時のメッセージを表示</span></div>
<div class="line"><a name="l00068"></a><span class="lineno">   68</span>&#160;        <span class="comment">//初期設定(利用する場合はコメントを外す)</span></div>
<div class="line"><a name="l00069"></a><span class="lineno">   69</span>&#160;        <span class="comment">//pointcloudlibrary.loadPLY(&quot;EV3COLOR.ply&quot;); //.plyファイルの読み込み.動作しない</span></div>
<div class="line"><a name="l00070"></a><span class="lineno">   70</span>&#160;        <span class="comment">//VideoWriter writer; //動画保存用 </span></div>
<div class="line"><a name="l00071"></a><span class="lineno">   71</span>&#160;</div>
<div class="line"><a name="l00072"></a><span class="lineno">   72</span>&#160;        <span class="comment">//ウインドウ名とファイル名の定義</span></div>
<div class="line"><a name="l00073"></a><span class="lineno">   73</span>&#160;        <span class="keyword">const</span> <span class="keywordtype">string</span> main_windowname = <span class="stringliteral">&quot;Current Image&quot;</span>; <span class="comment">//メインウインドウの名前をつけておく.(c31)</span></div>
<div class="line"><a name="l00074"></a><span class="lineno">   74</span>&#160;        <span class="keyword">const</span> <span class="keywordtype">string</span> backgroundimage_windowname = <span class="stringliteral">&quot;Background Image&quot;</span>; <span class="comment">//背景画像を表示するためのウインドウ名</span></div>
<div class="line"><a name="l00075"></a><span class="lineno">   75</span>&#160;        <span class="keyword">const</span> <span class="keywordtype">string</span> maskbinimage_windowname = <span class="stringliteral">&quot;Mask Image&quot;</span>; <span class="comment">//マスク画像を表示するためのウインドウ名</span></div>
<div class="line"><a name="l00076"></a><span class="lineno">   76</span>&#160;        <span class="comment">//const string outputVideoName = &quot;video.avi&quot;; //計測中の動画ファイル名(c39)</span></div>
<div class="line"><a name="l00077"></a><span class="lineno">   77</span>&#160;        <span class="keyword">const</span> <span class="keywordtype">string</span> cameraparameter_name = <span class="stringliteral">&quot;sourcedata/cameraParam.xml&quot;</span>; <span class="comment">//xmlファイル名の定義.カメラキャリブレーションによって得られたファイル名(c54)</span></div>
<div class="line"><a name="l00078"></a><span class="lineno">   78</span>&#160;        <span class="keyword">const</span> <span class="keywordtype">char</span>* datadirectory_name = <span class="stringliteral">&quot;data&quot;</span>; <span class="comment">//データ保存先のディレクトリ名</span></div>
<div class="line"><a name="l00079"></a><span class="lineno">   79</span>&#160;        <span class="keywordtype">char</span> base_dirname[<a class="code" href="stdafx_8h.html#a5d8ba032e2fdf7b2561f508164124f3e">NOC</a>]; <span class="comment">//出力先のdataと日付ディレクトリを合わせたもの(c86)</span></div>
<div class="line"><a name="l00080"></a><span class="lineno">   80</span>&#160;        <span class="keyword">const</span> <span class="keywordtype">string</span> cloudviewer_windowname = <span class="stringliteral">&quot;Cloud Viewer&quot;</span>; <span class="comment">//クラウドビューアーの名前の定義(c81)</span></div>
<div class="line"><a name="l00081"></a><span class="lineno">   81</span>&#160;        <span class="keyword">const</span> <span class="keywordtype">string</span> pclvisualizer_windowname = <span class="stringliteral">&quot;3D Viewer&quot;</span>;</div>
<div class="line"><a name="l00082"></a><span class="lineno">   82</span>&#160;        <span class="keyword">const</span> <span class="keywordtype">string</span> param_windowname = <span class="stringliteral">&quot;OpenCV Parameter Setting Window&quot;</span>; <span class="comment">//パラメータ調整用のウインドウ(c82)</span></div>
<div class="line"><a name="l00083"></a><span class="lineno">   83</span>&#160;</div>
<div class="line"><a name="l00084"></a><span class="lineno">   84</span>&#160;        kinect.<a class="code" href="class_kinect.html#a79f66d96dc810bf09a9d3cfe4f1f2671">initialize</a>(); <span class="comment">//Kinectの初期化</span></div>
<div class="line"><a name="l00085"></a><span class="lineno">   85</span>&#160;        sys.<a class="code" href="class_system.html#a2f2eb72dffbb596c530ca149fd8830cd">checkDirectory</a>(datadirectory_name); <span class="comment">//base_directoryが存在するか確認し,存在しなければ作成(c81)</span></div>
<div class="line"><a name="l00086"></a><span class="lineno">   86</span>&#160;        sys.<a class="code" href="class_system.html#a6353709a74c9a12a3b1c8e365db5f51e">makeDirectoryBasedDate</a>(); <span class="comment">//起動時刻をフォルダ名にしてフォルダを作成</span></div>
<div class="line"><a name="l00087"></a><span class="lineno">   87</span>&#160;        sprintf_s(base_dirname, <span class="stringliteral">&quot;%s/%s&quot;</span>, datadirectory_name, <a class="code" href="main_8cpp.html#abefb498e9a643f68bb3d37c22953ddad">directoryName</a>); <span class="comment">//保存先の基本となるディレクトリ名を定義(c86)</span></div>
<div class="line"><a name="l00088"></a><span class="lineno">   88</span>&#160;        </div>
<div class="line"><a name="l00089"></a><span class="lineno">   89</span>&#160;        <span class="comment">//動画を保存するために利用する(c40)</span></div>
<div class="line"><a name="l00090"></a><span class="lineno">   90</span>&#160;        <span class="comment">//writer = sys.outputVideo(&amp;outputVideoName); //動画を保存したいときはコメントをはずす.while文内のwriter &lt;&lt; imageのコメントも</span></div>
<div class="line"><a name="l00091"></a><span class="lineno">   91</span>&#160;</div>
<div class="line"><a name="l00092"></a><span class="lineno">   92</span>&#160;        imgproc.<a class="code" href="class_image_processing.html#a58862a0cb31373a4d1a9d6a89fdd8e6c">loadInternalCameraParameter</a>(cameraparameter_name); <span class="comment">//キャリブレーションを行うためのパラメータを取得(c79)</span></div>
<div class="line"><a name="l00093"></a><span class="lineno">   93</span>&#160;</div>
<div class="line"><a name="l00094"></a><span class="lineno">   94</span>&#160;        <span class="comment">//背景用に一度撮影(c75)</span></div>
<div class="line"><a name="l00095"></a><span class="lineno">   95</span>&#160;        cout &lt;&lt; <span class="stringliteral">&quot;Take a Background Image in &quot;</span> &lt;&lt; endl;</div>
<div class="line"><a name="l00096"></a><span class="lineno">   96</span>&#160;        sys.<a class="code" href="class_system.html#afce9f814824cbcb337abe755883f764c">countdownTimer</a>(3);</div>
<div class="line"><a name="l00097"></a><span class="lineno">   97</span>&#160;        system(<span class="stringliteral">&quot;cls&quot;</span>);</div>
<div class="line"><a name="l00098"></a><span class="lineno">   98</span>&#160;        DWORD ret = ::WaitForSingleObject(kinect.<a class="code" href="class_kinect.html#a975c1de77489cffc0728ae9817e127c2">streamEvent</a>, INFINITE); <span class="comment">//フレーム更新をイベントとして待つ</span></div>
<div class="line"><a name="l00099"></a><span class="lineno">   99</span>&#160;        ::ResetEvent(kinect.<a class="code" href="class_kinect.html#a975c1de77489cffc0728ae9817e127c2">streamEvent</a>); <span class="comment">//イベントが発生したら次のイベントに備えてリセット</span></div>
<div class="line"><a name="l00100"></a><span class="lineno">  100</span>&#160;        <span class="comment">//Kinectから画像を取得し,背景画像とする</span></div>
<div class="line"><a name="l00101"></a><span class="lineno">  101</span>&#160;        PlaySound(TEXT(<span class="stringliteral">&quot;sourcedata/shutter.wav&quot;</span>), NULL, (SND_ASYNC | SND_FILENAME)); <span class="comment">//音声ファイルを再生</span></div>
<div class="line"><a name="l00102"></a><span class="lineno">  102</span>&#160;        background_image = kinect.<a class="code" href="class_kinect.html#a00f185eb803cc53eb6273000913d9892">drawRGBImage</a>(<a class="code" href="main_8cpp.html#aabb27b8973575043030df51be47cd24a">image</a>); <span class="comment">//RGBカメラの処理</span></div>
<div class="line"><a name="l00103"></a><span class="lineno">  103</span>&#160;        background_image = imgproc.<a class="code" href="class_image_processing.html#a5fa5bf016a9d7f946919d0a34a47031c">getUndistortionImage</a>(background_image); <span class="comment">//キャリブレーション後の画像を今の背景画像とする</span></div>
<div class="line"><a name="l00104"></a><span class="lineno">  104</span>&#160;        imgproc.<a class="code" href="class_image_processing.html#a77b6912e1d1182d29a7c8a8c639bb685">showImage</a>(backgroundimage_windowname, background_image); <span class="comment">//背景画像を表示する</span></div>
<div class="line"><a name="l00105"></a><span class="lineno">  105</span>&#160;        cvtColor(background_image, background_gray_image, CV_BGR2GRAY); <span class="comment">//グレースケールに変換</span></div>
<div class="line"><a name="l00106"></a><span class="lineno">  106</span>&#160;        </div>
<div class="line"><a name="l00107"></a><span class="lineno">  107</span>&#160;        Sleep(1000);</div>
<div class="line"><a name="l00108"></a><span class="lineno">  108</span>&#160;</div>
<div class="line"><a name="l00109"></a><span class="lineno">  109</span>&#160;        <span class="comment">//プログラム開始の通知</span></div>
<div class="line"><a name="l00110"></a><span class="lineno">  110</span>&#160;        cout &lt;&lt; <span class="stringliteral">&quot;Process will Start in &quot;</span> &lt;&lt; endl;</div>
<div class="line"><a name="l00111"></a><span class="lineno">  111</span>&#160;        sys.<a class="code" href="class_system.html#afce9f814824cbcb337abe755883f764c">countdownTimer</a>(2); </div>
<div class="line"><a name="l00112"></a><span class="lineno">  112</span>&#160;        PlaySound(TEXT(<span class="stringliteral">&quot;sourcedata/shutter.wav&quot;</span>), NULL, (SND_ASYNC | SND_FILENAME)); <span class="comment">//音声ファイルを再生</span></div>
<div class="line"><a name="l00113"></a><span class="lineno">  113</span>&#160;        system(<span class="stringliteral">&quot;cls&quot;</span>); <span class="comment">//コンソール内の表示をリセット(c64)</span></div>
<div class="line"><a name="l00114"></a><span class="lineno">  114</span>&#160;</div>
<div class="line"><a name="l00115"></a><span class="lineno">  115</span>&#160;        pointcloudlibrary.<a class="code" href="class_point_cloud_library.html#a7c0232fc05e878dd58f64c1e6ed8e0b4">initializePCLVisualizer</a>(pclvisualizer_windowname); <span class="comment">//クラウドビジュアライザーの初期化</span></div>
<div class="line"><a name="l00116"></a><span class="lineno">  116</span>&#160;</div>
<div class="line"><a name="l00117"></a><span class="lineno">  117</span>&#160;        <span class="keywordflow">while</span> (!pointcloudlibrary.<a class="code" href="class_point_cloud_library.html#a00af6b104220e288f26c2d6f890ccd86">visualizer</a>-&gt;wasStopped() &amp;&amp; kinect.<a class="code" href="class_kinect.html#afe9df20df1f2538ac9ce8b61c409095a">key</a> != <span class="charliteral">&#39;q&#39;</span> &amp;&amp; !GetAsyncKeyState(<span class="charliteral">&#39;Q&#39;</span>)){ <span class="comment">//(c3).メインループ.1フレームごとの処理を繰り返し行う.(c63)CloudViewerが終了処理(&#39;q&#39;キーを入力)したらプログラムが終了する</span></div>
<div class="line"><a name="l00118"></a><span class="lineno">  118</span>&#160;            <span class="comment">//タイマー開始(c65)</span></div>
<div class="line"><a name="l00119"></a><span class="lineno">  119</span>&#160;            sys.<a class="code" href="class_system.html#ac62488af6f94b96aa37988c54e19074f">startTimer</a>();</div>
<div class="line"><a name="l00120"></a><span class="lineno">  120</span>&#160;</div>
<div class="line"><a name="l00121"></a><span class="lineno">  121</span>&#160;            <span class="comment">//setMouseCallback(mainwindow_name, onMouse, 0); //(c25).マウスコールバック関数をセット(c31)</span></div>
<div class="line"><a name="l00122"></a><span class="lineno">  122</span>&#160;</div>
<div class="line"><a name="l00123"></a><span class="lineno">  123</span>&#160;            DWORD ret = ::WaitForSingleObject(kinect.<a class="code" href="class_kinect.html#a975c1de77489cffc0728ae9817e127c2">streamEvent</a>, INFINITE); <span class="comment">//フレーム更新をイベントとして待つ</span></div>
<div class="line"><a name="l00124"></a><span class="lineno">  124</span>&#160;            ::ResetEvent(kinect.<a class="code" href="class_kinect.html#a975c1de77489cffc0728ae9817e127c2">streamEvent</a>); <span class="comment">//イベントが発生したら次のイベントに備えてリセット</span></div>
<div class="line"><a name="l00125"></a><span class="lineno">  125</span>&#160;</div>
<div class="line"><a name="l00126"></a><span class="lineno">  126</span>&#160;            <span class="comment">//===================================== 画像処理開始 =====================================</span></div>
<div class="line"><a name="l00127"></a><span class="lineno">  127</span>&#160;            <span class="comment">//Kinectから画像を取得し,画像処理を行う</span></div>
<div class="line"><a name="l00128"></a><span class="lineno">  128</span>&#160;            current_image = kinect.<a class="code" href="class_kinect.html#a00f185eb803cc53eb6273000913d9892">drawRGBImage</a>(<a class="code" href="main_8cpp.html#aabb27b8973575043030df51be47cd24a">image</a>); <span class="comment">//RGBカメラの処理</span></div>
<div class="line"><a name="l00129"></a><span class="lineno">  129</span>&#160;            current_image = imgproc.<a class="code" href="class_image_processing.html#a5fa5bf016a9d7f946919d0a34a47031c">getUndistortionImage</a>(current_image); <span class="comment">//Kinectのキャリブレーションを行い,キャリブレーション結果を適用する(c71)</span></div>
<div class="line"><a name="l00130"></a><span class="lineno">  130</span>&#160;            imgproc.<a class="code" href="class_image_processing.html#a77b6912e1d1182d29a7c8a8c639bb685">showImage</a>(main_windowname, current_image);</div>
<div class="line"><a name="l00131"></a><span class="lineno">  131</span>&#160;            <span class="comment">//flip(current_image, flip_image, 1);</span></div>
<div class="line"><a name="l00132"></a><span class="lineno">  132</span>&#160;            <span class="comment">//imgproc.showImage(&quot;Original - Flip&quot;, flip_image); //Kinectから取得した画像を表示</span></div>
<div class="line"><a name="l00133"></a><span class="lineno">  133</span>&#160;            </div>
<div class="line"><a name="l00134"></a><span class="lineno">  134</span>&#160;            <span class="comment">//タイヤも含めて前面の点群を取得する</span></div>
<div class="line"><a name="l00135"></a><span class="lineno">  135</span>&#160;            imgproc.<a class="code" href="class_image_processing.html#a36d146288e958736fd803fde49c064af">openCVSettingTrackbar</a>(maskbinimage_windowname);</div>
<div class="line"><a name="l00136"></a><span class="lineno">  136</span>&#160;            bin_image = imgproc.<a class="code" href="class_image_processing.html#ae7dc0c506adedb62687e15f0a1b1c7f0">getBackgroundSubstractionBinImage</a>(current_image, background_gray_image<span class="comment">/*, imgproc.th, imgproc.med, imgproc.cnt*/</span>);</div>
<div class="line"><a name="l00137"></a><span class="lineno">  137</span>&#160;            <span class="comment">//ユニット部だけ切り取る(c77)</span></div>
<div class="line"><a name="l00138"></a><span class="lineno">  138</span>&#160;            <span class="comment">//bin_image = imgproc.getUnitMask(bin_image);</span></div>
<div class="line"><a name="l00139"></a><span class="lineno">  139</span>&#160;            imgproc.<a class="code" href="class_image_processing.html#a77b6912e1d1182d29a7c8a8c639bb685">showImage</a>(maskbinimage_windowname, bin_image); <span class="comment">//確認用に切り取った画像を表示する</span></div>
<div class="line"><a name="l00140"></a><span class="lineno">  140</span>&#160;            <span class="comment">//===================================== 画像処理終了 =====================================</span></div>
<div class="line"><a name="l00141"></a><span class="lineno">  141</span>&#160;</div>
<div class="line"><a name="l00142"></a><span class="lineno">  142</span>&#160;            <span class="comment">//===================================== PCLで行う処理開始 =====================================</span></div>
<div class="line"><a name="l00143"></a><span class="lineno">  143</span>&#160;            <span class="comment">//ポイントクラウドの取得(c57)</span></div>
<div class="line"><a name="l00144"></a><span class="lineno">  144</span>&#160;            cloud = kinect.<a class="code" href="class_kinect.html#a26f88eae66c74699487869d10adb0945">getPointCloud</a>(bin_image); <span class="comment">//ポイントクラウドの取得(c57).切り取った画像をもとにする</span></div>
<div class="line"><a name="l00145"></a><span class="lineno">  145</span>&#160;            pointcloudlibrary.<a class="code" href="class_point_cloud_library.html#a1ee163b625dc199c44149db55c84bafd">flagChecker</a>(); <span class="comment">//各点群処理のフラグをチェックするメソッド(c64)</span></div>
<div class="line"><a name="l00146"></a><span class="lineno">  146</span>&#160;            cout &lt;&lt; <span class="stringliteral">&quot;==========================================================================================&quot;</span> &lt;&lt; endl;</div>
<div class="line"><a name="l00147"></a><span class="lineno">  147</span>&#160;            cout &lt;&lt; <span class="stringliteral">&quot;Original PointCloud Size\t=&gt;\t&quot;</span> &lt;&lt; cloud-&gt;size() &lt;&lt; endl;</div>
<div class="line"><a name="l00148"></a><span class="lineno">  148</span>&#160;</div>
<div class="line"><a name="l00149"></a><span class="lineno">  149</span>&#160;            <span class="comment">//namedWindow(&quot;PCL&quot;, CV_WINDOW_KEEPRATIO);</span></div>
<div class="line"><a name="l00150"></a><span class="lineno">  150</span>&#160;            <span class="comment">//createTrackbar(&quot;thresh&quot;, &quot;PCL&quot;, &amp;pointcloudlibrary.th,1000);</span></div>
<div class="line"><a name="l00151"></a><span class="lineno">  151</span>&#160;            <span class="comment">//createTrackbar(&quot;tol&quot;, &quot;PCL&quot;, &amp;pointcloudlibrary.tor, 1000);</span></div>
<div class="line"><a name="l00152"></a><span class="lineno">  152</span>&#160;</div>
<div class="line"><a name="l00153"></a><span class="lineno">  153</span>&#160;            <span class="comment">//PCLの処理</span></div>
<div class="line"><a name="l00154"></a><span class="lineno">  154</span>&#160;            <span class="keywordflow">if</span> (pointcloudlibrary.<a class="code" href="class_point_cloud_library.html#a29aa3ebced9106bfc070d557efdbdbcf">passthrough_flag</a> == <span class="keyword">true</span>){  <span class="comment">//外れ値除去(c59)</span></div>
<div class="line"><a name="l00155"></a><span class="lineno">  155</span>&#160;                cloud = pointcloudlibrary.<a class="code" href="class_point_cloud_library.html#a1381846354368f5925e3b4e7d649c902">passThroughFilter</a>(cloud, <span class="stringliteral">&quot;z&quot;</span>, 400, 30000); <span class="comment">//Kinectから取得した外れ値を除去(c60).与えた軸の中で自分が取得したい範囲の下限と上限を与えることでそれ以外を省く(c81)</span></div>
<div class="line"><a name="l00156"></a><span class="lineno">  156</span>&#160;                <span class="comment">//cloud = pointcloudlibrary.radiusOutlierRemoval(cloud); //半径を指定して外れ値を除去(c60)</span></div>
<div class="line"><a name="l00157"></a><span class="lineno">  157</span>&#160;            }</div>
<div class="line"><a name="l00158"></a><span class="lineno">  158</span>&#160;            <span class="keywordflow">if</span> (pointcloudlibrary.<a class="code" href="class_point_cloud_library.html#a6bb70f323f325576a505c45edd1f2ebe">downsampling_flag</a> == <span class="keyword">true</span>){   <span class="comment">//ダウンサンプリング処理(c59)</span></div>
<div class="line"><a name="l00159"></a><span class="lineno">  159</span>&#160;                <span class="comment">//cloud = pointcloudlibrary.downSamplingUsingVoxelGridFilter(pointcloudlibrary.cloud, 2.0f, 2.0f, 2.0f); //Default=all 0.003</span></div>
<div class="line"><a name="l00160"></a><span class="lineno">  160</span>&#160;                cloud = pointcloudlibrary.<a class="code" href="class_point_cloud_library.html#abf4a5840ba484d07a775c2c0c9e41a99">downSamplingUsingVoxelGridFilter</a>(cloud, 2.5f, 2.5f, 2.5f); <span class="comment">//Default=all 0.003</span></div>
<div class="line"><a name="l00161"></a><span class="lineno">  161</span>&#160;                <span class="comment">//pointcloudlibrary.cloud = pointcloudlibrary.downSamplingUsingVoxelGridFilter(pointcloudlibrary.cloud, 0.001, 0.001, 0.001); //Default=all 0.003</span></div>
<div class="line"><a name="l00162"></a><span class="lineno">  162</span>&#160;            }</div>
<div class="line"><a name="l00163"></a><span class="lineno">  163</span>&#160;            <span class="keywordflow">if</span> (pointcloudlibrary.<a class="code" href="class_point_cloud_library.html#a25f7509ec80771a394266cc98c117a9b">statisticaloutlierremoval_flag</a> == <span class="keyword">true</span>){</div>
<div class="line"><a name="l00164"></a><span class="lineno">  164</span>&#160;                cloud = pointcloudlibrary.<a class="code" href="class_point_cloud_library.html#a4f27c0dfaae2be0174d733e295d1bf36">removeOutlier</a>(cloud); <span class="comment">//統計的な外れ値除去(c60)</span></div>
<div class="line"><a name="l00165"></a><span class="lineno">  165</span>&#160;            }</div>
<div class="line"><a name="l00166"></a><span class="lineno">  166</span>&#160;            <span class="keywordflow">if</span> (pointcloudlibrary.<a class="code" href="class_point_cloud_library.html#a6bb70f323f325576a505c45edd1f2ebe">downsampling_flag</a> == <span class="keyword">true</span> &amp;&amp; pointcloudlibrary.<a class="code" href="class_point_cloud_library.html#a0745fdcbdae44048793a4dce33dfeb4f">mls_flag</a> == <span class="keyword">true</span>){  <span class="comment">//スムージング処理(c60)</span></div>
<div class="line"><a name="l00167"></a><span class="lineno">  167</span>&#160;                cloud = pointcloudlibrary.<a class="code" href="class_point_cloud_library.html#a2899ca3f4f6b2f8bce39524fba686e74">smoothingUsingMovingLeastSquare</a>(cloud, <span class="keyword">true</span>, <span class="keyword">true</span>, 2.5); <span class="comment">//0.002 &lt; radius &lt; ◯.小さいほど除去される</span></div>
<div class="line"><a name="l00168"></a><span class="lineno">  168</span>&#160;</div>
<div class="line"><a name="l00169"></a><span class="lineno">  169</span>&#160;            }</div>
<div class="line"><a name="l00170"></a><span class="lineno">  170</span>&#160;            <span class="keywordflow">else</span> <span class="keywordflow">if</span> (pointcloudlibrary.<a class="code" href="class_point_cloud_library.html#a6bb70f323f325576a505c45edd1f2ebe">downsampling_flag</a> == <span class="keyword">false</span> &amp;&amp; pointcloudlibrary.<a class="code" href="class_point_cloud_library.html#a0745fdcbdae44048793a4dce33dfeb4f">mls_flag</a> == <span class="keyword">true</span>){</div>
<div class="line"><a name="l00171"></a><span class="lineno">  171</span>&#160;                cout &lt;&lt; <span class="stringliteral">&quot;MLSを有効にするためには,ダウンサンプリングを有効にしてください&quot;</span> &lt;&lt; endl;</div>
<div class="line"><a name="l00172"></a><span class="lineno">  172</span>&#160;            }</div>
<div class="line"><a name="l00173"></a><span class="lineno">  173</span>&#160;            <span class="keywordflow">if</span> (pointcloudlibrary.<a class="code" href="class_point_cloud_library.html#a76c829d2feea3efcb84449a0c360ec43">extractplane_flag</a> == <span class="keyword">true</span>){   <span class="comment">//平面検出とクラスタリング(c61)</span></div>
<div class="line"><a name="l00174"></a><span class="lineno">  174</span>&#160;                cloud = pointcloudlibrary.<a class="code" href="class_point_cloud_library.html#a222d6644e01b9925d923d58eb7ea1522">getExtractPlaneAndClustering</a>(cloud, <span class="keyword">true</span>, 100,<span class="comment">/* 30, */</span><span class="keyword">false</span>, <span class="keyword">true</span>,<span class="comment">/* 50, */</span><span class="comment">/*350*/</span>150, <span class="comment">/*25000*/</span><span class="comment">/*20000*/</span>200000); <span class="comment">//Default=0.03(前処理なしの場合)</span></div>
<div class="line"><a name="l00175"></a><span class="lineno">  175</span>&#160;            }</div>
<div class="line"><a name="l00176"></a><span class="lineno">  176</span>&#160;</div>
<div class="line"><a name="l00177"></a><span class="lineno">  177</span>&#160;            <span class="comment">//平均座標を取得</span></div>
<div class="line"><a name="l00178"></a><span class="lineno">  178</span>&#160;            pointcloudlibrary.<a class="code" href="class_point_cloud_library.html#aa8b11edec4c6910b0a9ad5c5340dcfa2">centroid</a> = pointcloudlibrary.<a class="code" href="class_point_cloud_library.html#ad6ce847f97bd8bc4621a4c6a2c380b78">getCentroidCoordinate3d</a>(cloud); <span class="comment">//重心座標の計算</span></div>
<div class="line"><a name="l00179"></a><span class="lineno">  179</span>&#160;            </div>
<div class="line"><a name="l00180"></a><span class="lineno">  180</span>&#160;            <span class="comment">//平均座標のポイントクラウドを作成(c83)</span></div>
<div class="line"><a name="l00181"></a><span class="lineno">  181</span>&#160;            pcl::PointXYZ sphere; </div>
<div class="line"><a name="l00182"></a><span class="lineno">  182</span>&#160;            sphere.x = pointcloudlibrary.<a class="code" href="class_point_cloud_library.html#aa8b11edec4c6910b0a9ad5c5340dcfa2">centroid</a>.x; <span class="comment">//平均座標のx座標</span></div>
<div class="line"><a name="l00183"></a><span class="lineno">  183</span>&#160;            sphere.y = pointcloudlibrary.<a class="code" href="class_point_cloud_library.html#aa8b11edec4c6910b0a9ad5c5340dcfa2">centroid</a>.y; <span class="comment">//平均座標のy座標</span></div>
<div class="line"><a name="l00184"></a><span class="lineno">  184</span>&#160;            sphere.z = pointcloudlibrary.<a class="code" href="class_point_cloud_library.html#aa8b11edec4c6910b0a9ad5c5340dcfa2">centroid</a>.z; <span class="comment">//平均座標のz座標</span></div>
<div class="line"><a name="l00185"></a><span class="lineno">  185</span>&#160;</div>
<div class="line"><a name="l00186"></a><span class="lineno">  186</span>&#160;            lsm.<a class="code" href="class_least_square_method.html#a3b3f2a5f264c576367c3fb3e5d70dcf2">coefficient_plane</a> = lsm.<a class="code" href="class_least_square_method.html#a19559c76884063927045732a38fe043b">getCoefficient</a>(cloud); <span class="comment">//最小二乗法を行い平面の係数[a b c]&#39;を取得する(c78)</span></div>
<div class="line"><a name="l00187"></a><span class="lineno">  187</span>&#160;            lsm.<a class="code" href="class_least_square_method.html#a1ceee2394280e7d14411644de171782c">attitude_angle</a> = lsm.<a class="code" href="class_least_square_method.html#adf062bbbd7b380abc5f24d3a09d39e47">calcYawRollPitch</a>(lsm.<a class="code" href="class_least_square_method.html#a3b3f2a5f264c576367c3fb3e5d70dcf2">coefficient_plane</a>); <span class="comment">//姿勢角を取得(c78)</span></div>
<div class="line"><a name="l00188"></a><span class="lineno">  188</span>&#160;            ev3control.<a class="code" href="class_e_v3_control.html#a8ef402279dc736292ae0cf780dd3aed8">set6DoFEV3</a>(cloud, pointcloudlibrary.<a class="code" href="class_point_cloud_library.html#aa8b11edec4c6910b0a9ad5c5340dcfa2">centroid</a>, lsm.<a class="code" href="class_least_square_method.html#a1ceee2394280e7d14411644de171782c">attitude_angle</a>); <span class="comment">//6DoFをまとめる</span></div>
<div class="line"><a name="l00189"></a><span class="lineno">  189</span>&#160;            </div>
<div class="line"><a name="l00190"></a><span class="lineno">  190</span>&#160;            <span class="comment">//EV3の速度を計算(c85)</span></div>
<div class="line"><a name="l00191"></a><span class="lineno">  191</span>&#160;            <span class="comment">//ev3control.getVelocity(); //速度を計算</span></div>
<div class="line"><a name="l00192"></a><span class="lineno">  192</span>&#160;</div>
<div class="line"><a name="l00193"></a><span class="lineno">  193</span>&#160;            <span class="comment">//法線を計算(c84)</span></div>
<div class="line"><a name="l00194"></a><span class="lineno">  194</span>&#160;            <span class="comment">//cloud_normals = pointcloudlibrary.getSurfaceNormals(cloud); //法線を計算(c84)</span></div>
<div class="line"><a name="l00195"></a><span class="lineno">  195</span>&#160;            cout &lt;&lt; <span class="stringliteral">&quot;==========================================================================================&quot;</span> &lt;&lt; endl;</div>
<div class="line"><a name="l00196"></a><span class="lineno">  196</span>&#160;            <span class="comment">//===================================== PCLで行う処理終了 =====================================</span></div>
<div class="line"><a name="l00197"></a><span class="lineno">  197</span>&#160;</div>
<div class="line"><a name="l00198"></a><span class="lineno">  198</span>&#160;            <span class="comment">//===================================== 点群表示開始 =====================================</span></div>
<div class="line"><a name="l00199"></a><span class="lineno">  199</span>&#160;            <span class="comment">//Kinectから取得した点群を描画</span></div>
<div class="line"><a name="l00200"></a><span class="lineno">  200</span>&#160;            <span class="comment">//pointcloudlibrary.visualizer-&gt;addPointCloudNormals&lt;pcl::PointXYZRGB, pcl::Normal&gt;(cloud, cloud_normals, 30, 10, &quot;normals&quot;);</span></div>
<div class="line"><a name="l00201"></a><span class="lineno">  201</span>&#160;            pointcloudlibrary.<a class="code" href="class_point_cloud_library.html#a00af6b104220e288f26c2d6f890ccd86">visualizer</a>-&gt;addSphere(sphere, 10, 0.5, 0.0, 0.0, <span class="stringliteral">&quot;sphere&quot;</span>); <span class="comment">//平均座標に球を描画</span></div>
<div class="line"><a name="l00202"></a><span class="lineno">  202</span>&#160;            pointcloudlibrary.<a class="code" href="class_point_cloud_library.html#a00af6b104220e288f26c2d6f890ccd86">visualizer</a>-&gt;addPointCloud(cloud, <span class="stringliteral">&quot;show cloud&quot;</span>); <span class="comment">//点群を描画</span></div>
<div class="line"><a name="l00203"></a><span class="lineno">  203</span>&#160;            <span class="comment">//点群の表示</span></div>
<div class="line"><a name="l00204"></a><span class="lineno">  204</span>&#160;            pointcloudlibrary.<a class="code" href="class_point_cloud_library.html#a00af6b104220e288f26c2d6f890ccd86">visualizer</a>-&gt;spinOnce(); <span class="comment">//PCLVisualizerを描画</span></div>
<div class="line"><a name="l00205"></a><span class="lineno">  205</span>&#160;            <span class="comment">//PCLVisualizerに描画した点群を削除する</span></div>
<div class="line"><a name="l00206"></a><span class="lineno">  206</span>&#160;            pointcloudlibrary.<a class="code" href="class_point_cloud_library.html#a00af6b104220e288f26c2d6f890ccd86">visualizer</a>-&gt;removePointCloud(<span class="stringliteral">&quot;show cloud&quot;</span>);</div>
<div class="line"><a name="l00207"></a><span class="lineno">  207</span>&#160;            pointcloudlibrary.<a class="code" href="class_point_cloud_library.html#a00af6b104220e288f26c2d6f890ccd86">visualizer</a>-&gt;removeShape(<span class="stringliteral">&quot;sphere&quot;</span>);</div>
<div class="line"><a name="l00208"></a><span class="lineno">  208</span>&#160;            <span class="comment">//===================================== 点群表示終了 =====================================</span></div>
<div class="line"><a name="l00209"></a><span class="lineno">  209</span>&#160;            </div>
<div class="line"><a name="l00210"></a><span class="lineno">  210</span>&#160;            <span class="comment">//===================================== キー入力関連開始 =====================================</span></div>
<div class="line"><a name="l00211"></a><span class="lineno">  211</span>&#160;            <span class="comment">//終了のためのキー入力チェック兼表示のためのウェイトタイム</span></div>
<div class="line"><a name="l00212"></a><span class="lineno">  212</span>&#160;            kinect.<a class="code" href="class_kinect.html#afe9df20df1f2538ac9ce8b61c409095a">key</a> = waitKey(1); <span class="comment">//OpenCVのウインドウを表示し続ける</span></div>
<div class="line"><a name="l00213"></a><span class="lineno">  213</span>&#160;            <span class="comment">//キーが入力されていれば以下を実行する.GetAsyncKeyStateを利用することで</span></div>
<div class="line"><a name="l00214"></a><span class="lineno">  214</span>&#160;            <span class="keywordflow">if</span> (GetAsyncKeyState(<span class="charliteral">&#39;R&#39;</span>)){ <span class="comment">//Rキーの入力処理</span></div>
<div class="line"><a name="l00215"></a><span class="lineno">  215</span>&#160;                destroyAllWindows(); <span class="comment">//PCLまたは,OpenCV画面上で&#39;q&#39;キーが入力されたらOpenCVのウインドウを閉じて処理を終了(c66)</span></div>
<div class="line"><a name="l00216"></a><span class="lineno">  216</span>&#160;                sys.<a class="code" href="class_system.html#a3523df6c59162a85350b41a065efa3dd">removeDirectory</a>(); <span class="comment">//再計測を行う場合は,現在のデータは必要ないため削除</span></div>
<div class="line"><a name="l00217"></a><span class="lineno">  217</span>&#160;                save_count = 0;</div>
<div class="line"><a name="l00218"></a><span class="lineno">  218</span>&#160;                cout &lt;&lt; <span class="stringliteral">&quot;Data Removed.&quot;</span> &lt;&lt; endl;</div>
<div class="line"><a name="l00219"></a><span class="lineno">  219</span>&#160;                system(<span class="stringliteral">&quot;cls&quot;</span>); <span class="comment">//cmdをクリア</span></div>
<div class="line"><a name="l00220"></a><span class="lineno">  220</span>&#160;                cout &lt;&lt; <span class="stringliteral">&quot;RETRY&quot;</span> &lt;&lt; endl;</div>
<div class="line"><a name="l00221"></a><span class="lineno">  221</span>&#160;                <span class="keywordflow">goto</span> RETRY;</div>
<div class="line"><a name="l00222"></a><span class="lineno">  222</span>&#160;            }</div>
<div class="line"><a name="l00223"></a><span class="lineno">  223</span>&#160;            <span class="keywordflow">else</span> <span class="keywordflow">if</span> (GetAsyncKeyState(<span class="charliteral">&#39;P&#39;</span>)){ <span class="comment">//その時点のデータを保存する.複数回データを計測する際はプログラムを起動しなおす手間が省ける</span></div>
<div class="line"><a name="l00224"></a><span class="lineno">  224</span>&#160;                <span class="keywordtype">char</span> output_basedirpath[<a class="code" href="stdafx_8h.html#a5d8ba032e2fdf7b2561f508164124f3e">NOC</a>]; <span class="comment">//その時ごとの保存先のパスを作成する</span></div>
<div class="line"><a name="l00225"></a><span class="lineno">  225</span>&#160;                sprintf_s(output_basedirpath, <span class="stringliteral">&quot;%s/%02d&quot;</span>, base_dirname, save_count);</div>
<div class="line"><a name="l00226"></a><span class="lineno">  226</span>&#160;                sys.<a class="code" href="class_system.html#a4e8e357a7875f40a86cee5cfdad6b6ed">makeDirectory</a>(base_dirname, save_count); <span class="comment">//その時点でのディレクトリを作成</span></div>
<div class="line"><a name="l00227"></a><span class="lineno">  227</span>&#160;                imgproc.<a class="code" href="class_image_processing.html#ac2a3a10ecc09a2be23b2cc673cee1e0c">outputImageSelectDirectory</a>(save_count, output_basedirpath, <span class="stringliteral">&quot;current-image&quot;</span>, current_image); <span class="comment">//現在の画像を出力</span></div>
<div class="line"><a name="l00228"></a><span class="lineno">  228</span>&#160;                imgproc.<a class="code" href="class_image_processing.html#ac2a3a10ecc09a2be23b2cc673cee1e0c">outputImageSelectDirectory</a>(save_count, output_basedirpath, <span class="stringliteral">&quot;mask-image&quot;</span>, bin_image); <span class="comment">//作成したマスク画像を保存</span></div>
<div class="line"><a name="l00229"></a><span class="lineno">  229</span>&#160;                pointcloudlibrary.<a class="code" href="class_point_cloud_library.html#ac46bf4f8f9f819b8418e989fcb88f2e8">outputPointCloud</a>(save_count, output_basedirpath, <span class="stringliteral">&quot;pointcloud&quot;</span>, cloud); <span class="comment">//現在の点群を出力</span></div>
<div class="line"><a name="l00230"></a><span class="lineno">  230</span>&#160;                ev3control.<a class="code" href="class_e_v3_control.html#afce936155dcdef2787d31741b2bb1868">output6DoF</a>(save_count, output_basedirpath, <span class="stringliteral">&quot;6dof&quot;</span>, cloud); <span class="comment">//6DoF情報を保存</span></div>
<div class="line"><a name="l00231"></a><span class="lineno">  231</span>&#160;                ev3control.<a class="code" href="class_e_v3_control.html#a8308b078e6fce216a0b5e8e6a502e7ab">output6DoFContinuous</a>(base_dirname, <span class="stringliteral">&quot;6dof_continuous&quot;</span>, cloud); <span class="comment">//6DoF情報を続けて保存</span></div>
<div class="line"><a name="l00232"></a><span class="lineno">  232</span>&#160;                pointcloudlibrary.<a class="code" href="class_point_cloud_library.html#a450a5b18fa99c06b091fc68fe57bbea7">outputPointCloudPLY</a>(save_count, output_basedirpath, <span class="stringliteral">&quot;pointcloud&quot;</span>, cloud); <span class="comment">//現在の点群をply形式で保存</span></div>
<div class="line"><a name="l00233"></a><span class="lineno">  233</span>&#160;                draw.<a class="code" href="class_drawing.html#a98e39cb40a24d5909954c17fc1eaaa4f">gnuplotScriptEV3</a>(save_count, output_basedirpath, <span class="stringliteral">&quot;splot_ev3&quot;</span>, lsm.<a class="code" href="class_least_square_method.html#a3b3f2a5f264c576367c3fb3e5d70dcf2">coefficient_plane</a>); <span class="comment">//gnuplot用のスクリプト</span></div>
<div class="line"><a name="l00234"></a><span class="lineno">  234</span>&#160;                cout &lt;&lt; <span class="stringliteral">&quot;Save Current Data.&quot;</span> &lt;&lt; endl;</div>
<div class="line"><a name="l00235"></a><span class="lineno">  235</span>&#160;                ev3control.<a class="code" href="class_e_v3_control.html#abdd8dd38a39186e10013869fa4236605">save_flag</a> = <span class="keyword">true</span>; <span class="comment">//6DoF情報を出力するフラグをオンにする(c82)</span></div>
<div class="line"><a name="l00236"></a><span class="lineno">  236</span>&#160;                save_count++; <span class="comment">//保存した数をカウントする変数をインクリメントする</span></div>
<div class="line"><a name="l00237"></a><span class="lineno">  237</span>&#160;            }</div>
<div class="line"><a name="l00238"></a><span class="lineno">  238</span>&#160;            <span class="keywordflow">else</span> <span class="keywordflow">if</span> (GetAsyncKeyState(<span class="charliteral">&#39;L&#39;</span>)){ <span class="comment">//&#39;l&#39;が入力されたら.EV3の平均速度・ヨー角,軌道が欲しい時に入力する</span></div>
<div class="line"><a name="l00239"></a><span class="lineno">  239</span>&#160;                saveev3route_flag = <span class="keyword">true</span>; <span class="comment">//処理を実行するフラグをtrueにする</span></div>
<div class="line"><a name="l00240"></a><span class="lineno">  240</span>&#160;            }</div>
<div class="line"><a name="l00241"></a><span class="lineno">  241</span>&#160;            <span class="keywordflow">else</span> <span class="keywordflow">if</span> (GetAsyncKeyState(<span class="charliteral">&#39;H&#39;</span>)){ <span class="comment">//&#39;h&#39;が入力されたらヘルプを表示</span></div>
<div class="line"><a name="l00242"></a><span class="lineno">  242</span>&#160;                sys.<a class="code" href="class_system.html#af0fc7b12dc7342e2109a9aa3894b9bbe">showHelpMessage</a>(); <span class="comment">//ヘルプメッセージの表示</span></div>
<div class="line"><a name="l00243"></a><span class="lineno">  243</span>&#160;            }</div>
<div class="line"><a name="l00244"></a><span class="lineno">  244</span>&#160;            <span class="comment">//===================================== キー入力関連終了 =====================================</span></div>
<div class="line"><a name="l00245"></a><span class="lineno">  245</span>&#160;            </div>
<div class="line"><a name="l00246"></a><span class="lineno">  246</span>&#160;            <span class="comment">//===================================== 処理時間計測 =====================================</span></div>
<div class="line"><a name="l00247"></a><span class="lineno">  247</span>&#160;            sys.<a class="code" href="class_system.html#a178a7a315866961f208b6cab6c2d7e88">endTimer</a>(); <span class="comment">//タイマーを終了(c65)</span></div>
<div class="line"><a name="l00248"></a><span class="lineno">  248</span>&#160;            cout &lt;&lt; sys.<a class="code" href="class_system.html#a5f235cfc9736103d150e9dbeee82e543">getProcessTimeinMiliseconds</a>() &lt;&lt; <span class="stringliteral">&quot;[ms], &quot;</span> &lt;&lt; sys.<a class="code" href="class_system.html#a584a2b655a935ea404937cf8f11ab43c">getFrameRate</a>() &lt;&lt; <span class="stringliteral">&quot; fps&quot;</span> &lt;&lt; <span class="stringliteral">&quot;\n&quot;</span> &lt;&lt; endl;</div>
<div class="line"><a name="l00249"></a><span class="lineno">  249</span>&#160;            sys.<a class="code" href="class_system.html#aab7a0fe9f9f094ab4462a6f0e5765d92">sum_time</a> = sys.<a class="code" href="class_system.html#aab7a0fe9f9f094ab4462a6f0e5765d92">sum_time</a> + sys.<a class="code" href="class_system.html#a5f235cfc9736103d150e9dbeee82e543">getProcessTimeinMiliseconds</a>(); <span class="comment">//合計時間を計算</span></div>
<div class="line"><a name="l00250"></a><span class="lineno">  250</span>&#160;            <span class="comment">//===================================== 処理時間計測 =====================================</span></div>
<div class="line"><a name="l00251"></a><span class="lineno">  251</span>&#160;</div>
<div class="line"><a name="l00252"></a><span class="lineno">  252</span>&#160;            <span class="comment">//===================================== フラグがtrueのとき実行する処理の開始 =====================================</span></div>
<div class="line"><a name="l00253"></a><span class="lineno">  253</span>&#160;            <span class="comment">//&#39;l&#39;キーが入力されていれば,平均座標の軌道を追跡し続ける(c82)</span></div>
<div class="line"><a name="l00254"></a><span class="lineno">  254</span>&#160;            <span class="keywordflow">if</span> (saveev3route_flag == <span class="keyword">true</span>){ <span class="comment">//フラグがtrueであれば,平均座標の軌道を保存する(c82)</span></div>
<div class="line"><a name="l00255"></a><span class="lineno">  255</span>&#160;                <span class="comment">//outputdatafile.saveDataContinuously(sys.sum_time, ev3control.ev3_6dof, current);</span></div>
<div class="line"><a name="l00256"></a><span class="lineno">  256</span>&#160;                <span class="comment">//EV3の速度を計算(c85)</span></div>
<div class="line"><a name="l00257"></a><span class="lineno">  257</span>&#160;                ev3control.<a class="code" href="class_e_v3_control.html#a206f5da82bb48c5ddae1906b9206d773">getVelocityinSec</a>(sys.<a class="code" href="class_system.html#a5f235cfc9736103d150e9dbeee82e543">getProcessTimeinMiliseconds</a>()); <span class="comment">//速度を計算</span></div>
<div class="line"><a name="l00258"></a><span class="lineno">  258</span>&#160;                ev3control.<a class="code" href="class_e_v3_control.html#a2231f75bbba634cbb018ec9bf2de75a0">getAverageVelocityAndYaw</a>(); <span class="comment">//平均速度とヨー角を計算する</span></div>
<div class="line"><a name="l00259"></a><span class="lineno">  259</span>&#160;                ev3control.<a class="code" href="class_e_v3_control.html#a03d5b0e6707c302de24db329116f2142">outputEV3RouteContinuous</a>(base_dirname, <span class="stringliteral">&quot;ev3route&quot;</span>); <span class="comment">//EV3の軌道をファイルに出力</span></div>
<div class="line"><a name="l00260"></a><span class="lineno">  260</span>&#160;                ev3control.<a class="code" href="class_e_v3_control.html#a6bec8596128f406b0a90a6b206e1fd67">outputControlInformation</a>(sys.<a class="code" href="class_system.html#aab7a0fe9f9f094ab4462a6f0e5765d92">sum_time</a>, base_dirname, <span class="stringliteral">&quot;time-averagevandyaw&quot;</span>); <span class="comment">//時間と平均速度・平均ヨー角を出力</span></div>
<div class="line"><a name="l00261"></a><span class="lineno">  261</span>&#160;                ev3control.<a class="code" href="class_e_v3_control.html#a6bec8596128f406b0a90a6b206e1fd67">outputControlInformation</a>(); <span class="comment">//EV3の速度とヨー角をファイルに出力</span></div>
<div class="line"><a name="l00262"></a><span class="lineno">  262</span>&#160;            }</div>
<div class="line"><a name="l00263"></a><span class="lineno">  263</span>&#160;            <span class="comment">//===================================== フラグがtrueのとき実行する処理の終了 =====================================</span></div>
<div class="line"><a name="l00264"></a><span class="lineno">  264</span>&#160;</div>
<div class="line"><a name="l00265"></a><span class="lineno">  265</span>&#160;            <span class="comment">//system(&quot;cls&quot;); //コンソール内の表示をリセット(c64)</span></div>
<div class="line"><a name="l00266"></a><span class="lineno">  266</span>&#160;        }</div>
<div class="line"><a name="l00267"></a><span class="lineno">  267</span>&#160;</div>
<div class="line"><a name="l00268"></a><span class="lineno">  268</span>&#160;        <span class="comment">//===================================== 計測終了後処理開始 =====================================</span></div>
<div class="line"><a name="l00269"></a><span class="lineno">  269</span>&#160;        <span class="comment">//計測が終了したところ(PCL上, OpenCVウインドウ上, コンソール上で&#39;q&#39;が押されてたとき)</span></div>
<div class="line"><a name="l00270"></a><span class="lineno">  270</span>&#160;        destroyAllWindows(); <span class="comment">//PCLまたは,OpenCV画面上で&#39;q&#39;キーが入力されたらOpenCVのウインドウを閉じて処理を終了(c66)</span></div>
<div class="line"><a name="l00271"></a><span class="lineno">  271</span>&#160;        pointcloudlibrary.<a class="code" href="class_point_cloud_library.html#a00af6b104220e288f26c2d6f890ccd86">visualizer</a>-&gt;~PCLVisualizer(); <span class="comment">//PCLVisualizerの削除</span></div>
<div class="line"><a name="l00272"></a><span class="lineno">  272</span>&#160;        <span class="keywordflow">if</span> (saveev3route_flag == <span class="keyword">true</span>){ <span class="comment">//処理の間にlが押されていれば</span></div>
<div class="line"><a name="l00273"></a><span class="lineno">  273</span>&#160;            draw.<a class="code" href="class_drawing.html#ac6357919dea1fea41ae447837abcd16b">gnuplotScriptEV3Route</a>(base_dirname,<span class="stringliteral">&quot;splot_ev3route&quot;</span>); <span class="comment">//軌道をプロットするスクリプトを保存する</span></div>
<div class="line"><a name="l00274"></a><span class="lineno">  274</span>&#160;            draw.<a class="code" href="class_drawing.html#aa1da1f33436ebf7bfc05cf7453a6395e">gnuplotScriptTime2V</a>(); <span class="comment">//時間と速度をプロットするスクリプトを生成</span></div>
<div class="line"><a name="l00275"></a><span class="lineno">  275</span>&#160;            draw.<a class="code" href="class_drawing.html#a4f7e39ba7325c4311761db832d680407">gnuplotScriptTime2Yaw</a>(); <span class="comment">//時間とヨー角をプロットするスクリプトを生成</span></div>
<div class="line"><a name="l00276"></a><span class="lineno">  276</span>&#160;        }</div>
<div class="line"><a name="l00277"></a><span class="lineno">  277</span>&#160;</div>
<div class="line"><a name="l00278"></a><span class="lineno">  278</span>&#160;        <span class="comment">//データを保存するかの確認</span></div>
<div class="line"><a name="l00279"></a><span class="lineno">  279</span>&#160;        <span class="keywordflow">if</span> (saveev3route_flag == <span class="keyword">true</span> || ev3control.<a class="code" href="class_e_v3_control.html#abdd8dd38a39186e10013869fa4236605">save_flag</a> == <span class="keyword">true</span>){ <span class="comment">//データを保存するフラグがtrue(=データが保存されている)なら保存するかどうか確認する</span></div>
<div class="line"><a name="l00280"></a><span class="lineno">  280</span>&#160;            cout &lt;&lt; <span class="stringliteral">&quot;Save Data?&quot;</span> &lt;&lt; endl;</div>
<div class="line"><a name="l00281"></a><span class="lineno">  281</span>&#160;            <span class="keywordtype">int</span> checkNum = sys.<a class="code" href="class_system.html#ae6d222fb43f36cb2db3fb7d0bbed1c20">alternatives</a>(); <span class="comment">//&#39;1&#39;なら保存,&#39;0&#39;なら削除</span></div>
<div class="line"><a name="l00282"></a><span class="lineno">  282</span>&#160;            <span class="keywordflow">if</span> (checkNum == 1){ <span class="comment">//データを保存するなら</span></div>
<div class="line"><a name="l00283"></a><span class="lineno">  283</span>&#160;                sys.<a class="code" href="class_system.html#aba9d00b32578f2ea574bf67078db4aa7">endMessage</a>(checkNum);</div>
<div class="line"><a name="l00284"></a><span class="lineno">  284</span>&#160;            }</div>
<div class="line"><a name="l00285"></a><span class="lineno">  285</span>&#160;            <span class="keywordflow">else</span>{ <span class="comment">//データを削除するなら</span></div>
<div class="line"><a name="l00286"></a><span class="lineno">  286</span>&#160;                sys.<a class="code" href="class_system.html#a3523df6c59162a85350b41a065efa3dd">removeDirectory</a>(); <span class="comment">//ディレクトリの削除</span></div>
<div class="line"><a name="l00287"></a><span class="lineno">  287</span>&#160;                sys.<a class="code" href="class_system.html#aba9d00b32578f2ea574bf67078db4aa7">endMessage</a>(checkNum);</div>
<div class="line"><a name="l00288"></a><span class="lineno">  288</span>&#160;            }</div>
<div class="line"><a name="l00289"></a><span class="lineno">  289</span>&#160;        }</div>
<div class="line"><a name="l00290"></a><span class="lineno">  290</span>&#160;        <span class="keywordflow">else</span>{ <span class="comment">//データが一度も保存されていなければ確認せずにディレクトリを削除</span></div>
<div class="line"><a name="l00291"></a><span class="lineno">  291</span>&#160;            sys.<a class="code" href="class_system.html#a3523df6c59162a85350b41a065efa3dd">removeDirectory</a>(); <span class="comment">//ディレクトリの削除</span></div>
<div class="line"><a name="l00292"></a><span class="lineno">  292</span>&#160;            sys.<a class="code" href="class_system.html#aba9d00b32578f2ea574bf67078db4aa7">endMessage</a>();</div>
<div class="line"><a name="l00293"></a><span class="lineno">  293</span>&#160;        }</div>
<div class="line"><a name="l00294"></a><span class="lineno">  294</span>&#160;        <span class="comment">//===================================== 計測終了後処理終了 =====================================</span></div>
<div class="line"><a name="l00295"></a><span class="lineno">  295</span>&#160;</div>
<div class="line"><a name="l00296"></a><span class="lineno">  296</span>&#160;    }</div>
<div class="line"><a name="l00297"></a><span class="lineno">  297</span>&#160;    <span class="keywordflow">catch</span> (exception&amp; ex){ <span class="comment">//例外処理</span></div>
<div class="line"><a name="l00298"></a><span class="lineno">  298</span>&#160;        cout &lt;&lt; ex.what() &lt;&lt; endl;</div>
<div class="line"><a name="l00299"></a><span class="lineno">  299</span>&#160;        destroyAllWindows(); <span class="comment">//OpenCVで作成したウインドウを全て削除する(c35)</span></div>
<div class="line"><a name="l00300"></a><span class="lineno">  300</span>&#160;        <span class="comment">//pointcloudlibrary.viewer-&gt;~CloudViewer(); //クラウドビューアーの削除</span></div>
<div class="line"><a name="l00301"></a><span class="lineno">  301</span>&#160;        pointcloudlibrary.<a class="code" href="class_point_cloud_library.html#a00af6b104220e288f26c2d6f890ccd86">visualizer</a>-&gt;~PCLVisualizer(); <span class="comment">//PCLVisualizerの削除</span></div>
<div class="line"><a name="l00302"></a><span class="lineno">  302</span>&#160;        <span class="comment">//異常終了した時はデータを保存する必要がないので削除</span></div>
<div class="line"><a name="l00303"></a><span class="lineno">  303</span>&#160;        sys.<a class="code" href="class_system.html#a3523df6c59162a85350b41a065efa3dd">removeDirectory</a>();</div>
<div class="line"><a name="l00304"></a><span class="lineno">  304</span>&#160;        cout &lt;&lt; <span class="stringliteral">&quot;Data Removed.&quot;</span> &lt;&lt; endl;</div>
<div class="line"><a name="l00305"></a><span class="lineno">  305</span>&#160;        <span class="keywordflow">return</span> -1;</div>
<div class="line"><a name="l00306"></a><span class="lineno">  306</span>&#160;    }</div>
<div class="line"><a name="l00307"></a><span class="lineno">  307</span>&#160;    <span class="keywordflow">return</span> 0;</div>
<div class="line"><a name="l00308"></a><span class="lineno">  308</span>&#160;}</div>
<div class="ttc" id="class_system_html_ac62488af6f94b96aa37988c54e19074f"><div class="ttname"><a href="class_system.html#ac62488af6f94b96aa37988c54e19074f">System::startTimer</a></div><div class="ttdeci">void startTimer()</div><div class="ttdoc">タイマーを開始(c65) </div><div class="ttdef"><b>Definition:</b> <a href="_system_8cpp_source.html#l00157">System.cpp:157</a></div></div>
<div class="ttc" id="class_image_processing_html_a77b6912e1d1182d29a7c8a8c639bb685"><div class="ttname"><a href="class_image_processing.html#a77b6912e1d1182d29a7c8a8c639bb685">ImageProcessing::showImage</a></div><div class="ttdeci">void showImage(string windowName, Mat &amp;input_image)</div><div class="ttdoc">ウインドウの名前を引数に追加(c31).Matの表示(c17) </div><div class="ttdef"><b>Definition:</b> <a href="_image_processing_8cpp_source.html#l00036">ImageProcessing.cpp:36</a></div></div>
<div class="ttc" id="class_kinect_html_a26f88eae66c74699487869d10adb0945"><div class="ttname"><a href="class_kinect.html#a26f88eae66c74699487869d10adb0945">Kinect::getPointCloud</a></div><div class="ttdeci">pcl::PointCloud&lt; pcl::PointXYZRGB &gt;::Ptr getPointCloud(Mat &amp;Mat_image)</div><div class="ttdoc">Depthカメラの処理(c57) </div><div class="ttdef"><b>Definition:</b> <a href="_kinect_8cpp_source.html#l00109">Kinect.cpp:109</a></div></div>
<div class="ttc" id="class_point_cloud_library_html_ad6ce847f97bd8bc4621a4c6a2c380b78"><div class="ttname"><a href="class_point_cloud_library.html#ad6ce847f97bd8bc4621a4c6a2c380b78">PointCloudLibrary::getCentroidCoordinate3d</a></div><div class="ttdeci">Point3d getCentroidCoordinate3d(pcl::PointCloud&lt; pcl::PointXYZRGB &gt;::Ptr &amp;inputPointCloud)</div><div class="ttdoc">取得した点群の平均座標を取得するメソッド </div><div class="ttdef"><b>Definition:</b> <a href="_point_cloud_library_8cpp_source.html#l00331">PointCloudLibrary.cpp:331</a></div></div>
<div class="ttc" id="class_system_html_a6353709a74c9a12a3b1c8e365db5f51e"><div class="ttname"><a href="class_system.html#a6353709a74c9a12a3b1c8e365db5f51e">System::makeDirectoryBasedDate</a></div><div class="ttdeci">void makeDirectoryBasedDate()</div><div class="ttdoc">日付に基づいたディレクトリの作成 </div><div class="ttdef"><b>Definition:</b> <a href="_system_8cpp_source.html#l00247">System.cpp:247</a></div></div>
<div class="ttc" id="class_drawing_html_a4f7e39ba7325c4311761db832d680407"><div class="ttname"><a href="class_drawing.html#a4f7e39ba7325c4311761db832d680407">Drawing::gnuplotScriptTime2Yaw</a></div><div class="ttdeci">void gnuplotScriptTime2Yaw()</div><div class="ttdoc">時間とヨー角のプロット </div><div class="ttdef"><b>Definition:</b> <a href="_drawing_8cpp_source.html#l00110">Drawing.cpp:110</a></div></div>
<div class="ttc" id="class_e_v3_control_html_afce936155dcdef2787d31741b2bb1868"><div class="ttname"><a href="class_e_v3_control.html#afce936155dcdef2787d31741b2bb1868">EV3Control::output6DoF</a></div><div class="ttdeci">void output6DoF(int save_count, char *original_dirpath, char *output_filename, pcl::PointCloud&lt; pcl::PointXYZRGB &gt;::Ptr &amp;outputPointCloud)</div><div class="ttdoc">現フレームの6DoF情報をファイルに出力する </div><div class="ttdef"><b>Definition:</b> <a href="_e_v3_control_8cpp_source.html#l00150">EV3Control.cpp:150</a></div></div>
<div class="ttc" id="class_kinect_html_a00f185eb803cc53eb6273000913d9892"><div class="ttname"><a href="class_kinect.html#a00f185eb803cc53eb6273000913d9892">Kinect::drawRGBImage</a></div><div class="ttdeci">Mat drawRGBImage(Mat &amp;image)</div><div class="ttdoc">RGBカメラの処理 </div><div class="ttdef"><b>Definition:</b> <a href="_kinect_8cpp_source.html#l00081">Kinect.cpp:81</a></div></div>
<div class="ttc" id="class_system_html_a4e8e357a7875f40a86cee5cfdad6b6ed"><div class="ttname"><a href="class_system.html#a4e8e357a7875f40a86cee5cfdad6b6ed">System::makeDirectory</a></div><div class="ttdeci">void makeDirectory(char *original_path, int create_dirnum)</div><div class="ttdoc">指定したディレクトリ以下に新しいディレクトリを作成する( </div><div class="ttdef"><b>Definition:</b> <a href="_system_8cpp_source.html#l00266">System.cpp:266</a></div></div>
<div class="ttc" id="class_drawing_html_ac6357919dea1fea41ae447837abcd16b"><div class="ttname"><a href="class_drawing.html#ac6357919dea1fea41ae447837abcd16b">Drawing::gnuplotScriptEV3Route</a></div><div class="ttdeci">void gnuplotScriptEV3Route(char *original_dirpath, char *output_filename)</div><div class="ttdoc">EV3の軌道をプロットするためのスクリプト </div><div class="ttdef"><b>Definition:</b> <a href="_drawing_8cpp_source.html#l00072">Drawing.cpp:72</a></div></div>
<div class="ttc" id="_system_8hpp_html"><div class="ttname"><a href="_system_8hpp.html">System.hpp</a></div></div>
<div class="ttc" id="class_e_v3_control_html_a03d5b0e6707c302de24db329116f2142"><div class="ttname"><a href="class_e_v3_control.html#a03d5b0e6707c302de24db329116f2142">EV3Control::outputEV3RouteContinuous</a></div><div class="ttdeci">void outputEV3RouteContinuous(char *original_dirpath, char *output_filename)</div><div class="ttdoc">EV3の走行軌道を保存する </div><div class="ttdef"><b>Definition:</b> <a href="_e_v3_control_8cpp_source.html#l00189">EV3Control.cpp:189</a></div></div>
<div class="ttc" id="_least_square_method_8hpp_html"><div class="ttname"><a href="_least_square_method_8hpp.html">LeastSquareMethod.hpp</a></div></div>
<div class="ttc" id="class_point_cloud_library_html_a29aa3ebced9106bfc070d557efdbdbcf"><div class="ttname"><a href="class_point_cloud_library.html#a29aa3ebced9106bfc070d557efdbdbcf">PointCloudLibrary::passthrough_flag</a></div><div class="ttdeci">bool passthrough_flag</div><div class="ttdoc">パススルーフィルターを用いるかどうかのフラグ </div><div class="ttdef"><b>Definition:</b> <a href="_point_cloud_library_8hpp_source.html#l00054">PointCloudLibrary.hpp:54</a></div></div>
<div class="ttc" id="class_system_html_ae6d222fb43f36cb2db3fb7d0bbed1c20"><div class="ttname"><a href="class_system.html#ae6d222fb43f36cb2db3fb7d0bbed1c20">System::alternatives</a></div><div class="ttdeci">int alternatives()</div><div class="ttdoc">数字の入力をチェックする </div><div class="ttdef"><b>Definition:</b> <a href="_system_8cpp_source.html#l00295">System.cpp:295</a></div></div>
<div class="ttc" id="class_point_cloud_library_html_a00af6b104220e288f26c2d6f890ccd86"><div class="ttname"><a href="class_point_cloud_library.html#a00af6b104220e288f26c2d6f890ccd86">PointCloudLibrary::visualizer</a></div><div class="ttdeci">pcl::visualization::PCLVisualizer * visualizer</div><div class="ttdoc">PCL Visualizer. </div><div class="ttdef"><b>Definition:</b> <a href="_point_cloud_library_8hpp_source.html#l00048">PointCloudLibrary.hpp:48</a></div></div>
<div class="ttc" id="class_kinect_html_a79f66d96dc810bf09a9d3cfe4f1f2671"><div class="ttname"><a href="class_kinect.html#a79f66d96dc810bf09a9d3cfe4f1f2671">Kinect::initialize</a></div><div class="ttdeci">void initialize()</div><div class="ttdoc">Kinectの初期化 </div><div class="ttdef"><b>Definition:</b> <a href="_kinect_8cpp_source.html#l00058">Kinect.cpp:58</a></div></div>
<div class="ttc" id="class_image_processing_html_ac2a3a10ecc09a2be23b2cc673cee1e0c"><div class="ttname"><a href="class_image_processing.html#ac2a3a10ecc09a2be23b2cc673cee1e0c">ImageProcessing::outputImageSelectDirectory</a></div><div class="ttdeci">void outputImageSelectDirectory(int save_count, char *original_dirpath, char *save_filename, Mat &amp;output_image)</div><div class="ttdoc">メソッドImageProcessing::outputImageSelectDirectory().出力したディレクトリにファイルを出力するメソッド </div><div class="ttdef"><b>Definition:</b> <a href="_image_processing_8cpp_source.html#l00266">ImageProcessing.cpp:266</a></div></div>
<div class="ttc" id="class_image_processing_html"><div class="ttname"><a href="class_image_processing.html">ImageProcessing</a></div><div class="ttdoc">画像処理用のクラス </div><div class="ttdef"><b>Definition:</b> <a href="_image_processing_8hpp_source.html#l00019">ImageProcessing.hpp:19</a></div></div>
<div class="ttc" id="class_point_cloud_library_html"><div class="ttname"><a href="class_point_cloud_library.html">PointCloudLibrary</a></div><div class="ttdoc">点群処理を行うクラス </div><div class="ttdef"><b>Definition:</b> <a href="_point_cloud_library_8hpp_source.html#l00019">PointCloudLibrary.hpp:19</a></div></div>
<div class="ttc" id="class_system_html_a15325f6b106b804af8432d102945535c"><div class="ttname"><a href="class_system.html#a15325f6b106b804af8432d102945535c">System::startMessage</a></div><div class="ttdeci">void startMessage()</div><div class="ttdoc">プログラム開始時のメッセージを表示(c26) </div><div class="ttdef"><b>Definition:</b> <a href="_system_8cpp_source.html#l00036">System.cpp:36</a></div></div>
<div class="ttc" id="main_8cpp_html_aabb27b8973575043030df51be47cd24a"><div class="ttname"><a href="main_8cpp.html#aabb27b8973575043030df51be47cd24a">image</a></div><div class="ttdeci">Mat image</div><div class="ttdoc">RGB画像格納用の変数 </div><div class="ttdef"><b>Definition:</b> <a href="main_8cpp_source.html#l00020">main.cpp:20</a></div></div>
<div class="ttc" id="class_system_html_a178a7a315866961f208b6cab6c2d7e88"><div class="ttname"><a href="class_system.html#a178a7a315866961f208b6cab6c2d7e88">System::endTimer</a></div><div class="ttdeci">void endTimer()</div><div class="ttdoc">タイマーを終了(c65) </div><div class="ttdef"><b>Definition:</b> <a href="_system_8cpp_source.html#l00168">System.cpp:168</a></div></div>
<div class="ttc" id="class_point_cloud_library_html_ac46bf4f8f9f819b8418e989fcb88f2e8"><div class="ttname"><a href="class_point_cloud_library.html#ac46bf4f8f9f819b8418e989fcb88f2e8">PointCloudLibrary::outputPointCloud</a></div><div class="ttdeci">void outputPointCloud(int save_count, char *original_dirpath, char *output_filename, pcl::PointCloud&lt; pcl::PointXYZRGB &gt;::Ptr &amp;outputPointCloud)</div><div class="ttdoc">点群を出力するメソッド </div><div class="ttdef"><b>Definition:</b> <a href="_point_cloud_library_8cpp_source.html#l00404">PointCloudLibrary.cpp:404</a></div></div>
<div class="ttc" id="class_e_v3_control_html_a8308b078e6fce216a0b5e8e6a502e7ab"><div class="ttname"><a href="class_e_v3_control.html#a8308b078e6fce216a0b5e8e6a502e7ab">EV3Control::output6DoFContinuous</a></div><div class="ttdeci">void output6DoFContinuous(char *original_dirpath, char *output_filename, pcl::PointCloud&lt; pcl::PointXYZRGB &gt;::Ptr &amp;outputPointCloud)</div><div class="ttdoc">キーを入力したときの6DoF情報を連続してcsv形式で保存する </div><div class="ttdef"><b>Definition:</b> <a href="_e_v3_control_8cpp_source.html#l00169">EV3Control.cpp:169</a></div></div>
<div class="ttc" id="class_image_processing_html_ae7dc0c506adedb62687e15f0a1b1c7f0"><div class="ttname"><a href="class_image_processing.html#ae7dc0c506adedb62687e15f0a1b1c7f0">ImageProcessing::getBackgroundSubstractionBinImage</a></div><div class="ttdeci">Mat getBackgroundSubstractionBinImage(Mat &amp;current_image, Mat &amp;backgound_gray_image)</div><div class="ttdoc">背景差分によって得られた二値画像(c75) </div><div class="ttdef"><b>Definition:</b> <a href="_image_processing_8cpp_source.html#l00130">ImageProcessing.cpp:130</a></div></div>
<div class="ttc" id="class_point_cloud_library_html_a4f27c0dfaae2be0174d733e295d1bf36"><div class="ttname"><a href="class_point_cloud_library.html#a4f27c0dfaae2be0174d733e295d1bf36">PointCloudLibrary::removeOutlier</a></div><div class="ttdeci">pcl::PointCloud&lt; pcl::PointXYZRGB &gt;::Ptr removeOutlier(pcl::PointCloud&lt; pcl::PointXYZRGB &gt;::Ptr &amp;inputPointCloud)</div><div class="ttdoc">外れ値を除去するメソッド </div><div class="ttdef"><b>Definition:</b> <a href="_point_cloud_library_8cpp_source.html#l00127">PointCloudLibrary.cpp:127</a></div></div>
<div class="ttc" id="main_8cpp_html_ae66f6b31b5ad750f1fe042a706a4e3d4"><div class="ttname"><a href="main_8cpp.html#ae66f6b31b5ad750f1fe042a706a4e3d4">main</a></div><div class="ttdeci">int main()</div><div class="ttdoc">関数main() </div><div class="ttdef"><b>Definition:</b> <a href="main_8cpp_source.html#l00036">main.cpp:36</a></div></div>
<div class="ttc" id="class_system_html_a584a2b655a935ea404937cf8f11ab43c"><div class="ttname"><a href="class_system.html#a584a2b655a935ea404937cf8f11ab43c">System::getFrameRate</a></div><div class="ttdeci">double getFrameRate()</div><div class="ttdoc">フレームレートを取得.startTimer()とendTimer()が実行されていることが前提(c65) </div><div class="ttdef"><b>Definition:</b> <a href="_system_8cpp_source.html#l00209">System.cpp:209</a></div></div>
<div class="ttc" id="class_point_cloud_library_html_a6bb70f323f325576a505c45edd1f2ebe"><div class="ttname"><a href="class_point_cloud_library.html#a6bb70f323f325576a505c45edd1f2ebe">PointCloudLibrary::downsampling_flag</a></div><div class="ttdeci">bool downsampling_flag</div><div class="ttdoc">ダウンサンプリングを行うかどうかのフラグ </div><div class="ttdef"><b>Definition:</b> <a href="_point_cloud_library_8hpp_source.html#l00055">PointCloudLibrary.hpp:55</a></div></div>
<div class="ttc" id="class_point_cloud_library_html_a0745fdcbdae44048793a4dce33dfeb4f"><div class="ttname"><a href="class_point_cloud_library.html#a0745fdcbdae44048793a4dce33dfeb4f">PointCloudLibrary::mls_flag</a></div><div class="ttdeci">bool mls_flag</div><div class="ttdoc">スムージングを行うかどうかのフラグ </div><div class="ttdef"><b>Definition:</b> <a href="_point_cloud_library_8hpp_source.html#l00057">PointCloudLibrary.hpp:57</a></div></div>
<div class="ttc" id="main_8cpp_html_a71a276a0dc4ef6aa35740b58f10bcb39"><div class="ttname"><a href="main_8cpp.html#a71a276a0dc4ef6aa35740b58f10bcb39">selectObject</a></div><div class="ttdeci">bool selectObject</div><div class="ttdoc">オブジェクト選択 </div><div class="ttdef"><b>Definition:</b> <a href="main_8cpp_source.html#l00025">main.cpp:25</a></div></div>
<div class="ttc" id="class_kinect_html_afe9df20df1f2538ac9ce8b61c409095a"><div class="ttname"><a href="class_kinect.html#afe9df20df1f2538ac9ce8b61c409095a">Kinect::key</a></div><div class="ttdeci">int key</div><div class="ttdoc">ウィンドウ表示のウェイトタイム格納変数 </div><div class="ttdef"><b>Definition:</b> <a href="_kinect_8hpp_source.html#l00053">Kinect.hpp:53</a></div></div>
<div class="ttc" id="class_least_square_method_html"><div class="ttname"><a href="class_least_square_method.html">LeastSquareMethod</a></div><div class="ttdoc">最小二乗法を行うクラス </div><div class="ttdef"><b>Definition:</b> <a href="_least_square_method_8hpp_source.html#l00019">LeastSquareMethod.hpp:19</a></div></div>
<div class="ttc" id="class_point_cloud_library_html_a450a5b18fa99c06b091fc68fe57bbea7"><div class="ttname"><a href="class_point_cloud_library.html#a450a5b18fa99c06b091fc68fe57bbea7">PointCloudLibrary::outputPointCloudPLY</a></div><div class="ttdeci">void outputPointCloudPLY(int save_count, char *original_dirpath, char *output_filename, pcl::PointCloud&lt; pcl::PointXYZRGB &gt;::Ptr &amp;outputPointCloud)</div><div class="ttdoc">点群をply形式で保存するメソッド </div><div class="ttdef"><b>Definition:</b> <a href="_point_cloud_library_8cpp_source.html#l00425">PointCloudLibrary.cpp:425</a></div></div>
<div class="ttc" id="class_point_cloud_library_html_a1381846354368f5925e3b4e7d649c902"><div class="ttname"><a href="class_point_cloud_library.html#a1381846354368f5925e3b4e7d649c902">PointCloudLibrary::passThroughFilter</a></div><div class="ttdeci">pcl::PointCloud&lt; pcl::PointXYZRGB &gt;::Ptr passThroughFilter(pcl::PointCloud&lt; pcl::PointXYZRGB &gt;::Ptr &amp;inputPointCloud, char *axis, float min, float max)</div><div class="ttdoc">パススルーフィルタ.zの値の距離に応じてカット可能 </div><div class="ttdef"><b>Definition:</b> <a href="_point_cloud_library_8cpp_source.html#l00106">PointCloudLibrary.cpp:106</a></div></div>
<div class="ttc" id="class_e_v3_control_html_a2231f75bbba634cbb018ec9bf2de75a0"><div class="ttname"><a href="class_e_v3_control.html#a2231f75bbba634cbb018ec9bf2de75a0">EV3Control::getAverageVelocityAndYaw</a></div><div class="ttdeci">void getAverageVelocityAndYaw()</div><div class="ttdoc">平均の速度とヨー角を計算する </div><div class="ttdef"><b>Definition:</b> <a href="_e_v3_control_8cpp_source.html#l00089">EV3Control.cpp:89</a></div></div>
<div class="ttc" id="class_e_v3_control_html_a6bec8596128f406b0a90a6b206e1fd67"><div class="ttname"><a href="class_e_v3_control.html#a6bec8596128f406b0a90a6b206e1fd67">EV3Control::outputControlInformation</a></div><div class="ttdeci">void outputControlInformation(double sumtime_ms, char *original_dirpath, char *output_filename)</div><div class="ttdoc">EV3の制御情報を出力する </div><div class="ttdef"><b>Definition:</b> <a href="_e_v3_control_8cpp_source.html#l00208">EV3Control.cpp:208</a></div></div>
<div class="ttc" id="_e_v3_control_8hpp_html"><div class="ttname"><a href="_e_v3_control_8hpp.html">EV3Control.hpp</a></div></div>
<div class="ttc" id="class_system_html_aab7a0fe9f9f094ab4462a6f0e5765d92"><div class="ttname"><a href="class_system.html#aab7a0fe9f9f094ab4462a6f0e5765d92">System::sum_time</a></div><div class="ttdeci">double sum_time</div><div class="ttdoc">処理の合計時間 </div><div class="ttdef"><b>Definition:</b> <a href="_system_8hpp_source.html#l00046">System.hpp:46</a></div></div>
<div class="ttc" id="class_drawing_html"><div class="ttname"><a href="class_drawing.html">Drawing</a></div><div class="ttdoc">経路描画用のクラス </div><div class="ttdef"><b>Definition:</b> <a href="_drawing_8hpp_source.html#l00019">Drawing.hpp:19</a></div></div>
<div class="ttc" id="_drawing_8hpp_html"><div class="ttname"><a href="_drawing_8hpp.html">Drawing.hpp</a></div></div>
<div class="ttc" id="class_e_v3_control_html_abdd8dd38a39186e10013869fa4236605"><div class="ttname"><a href="class_e_v3_control.html#abdd8dd38a39186e10013869fa4236605">EV3Control::save_flag</a></div><div class="ttdeci">bool save_flag</div><div class="ttdoc">6DoF情報を出力するかチェックするためのフラグ </div><div class="ttdef"><b>Definition:</b> <a href="_e_v3_control_8hpp_source.html#l00048">EV3Control.hpp:48</a></div></div>
<div class="ttc" id="main_8cpp_html_abefb498e9a643f68bb3d37c22953ddad"><div class="ttname"><a href="main_8cpp.html#abefb498e9a643f68bb3d37c22953ddad">directoryName</a></div><div class="ttdeci">char directoryName[NOC]</div><div class="ttdoc">フォルダ名 </div><div class="ttdef"><b>Definition:</b> <a href="main_8cpp_source.html#l00022">main.cpp:22</a></div></div>
<div class="ttc" id="class_system_html_a3523df6c59162a85350b41a065efa3dd"><div class="ttname"><a href="class_system.html#a3523df6c59162a85350b41a065efa3dd">System::removeDirectory</a></div><div class="ttdeci">void removeDirectory()</div><div class="ttdoc">取得したデータが不要だった場合ディレクトリを削除する </div><div class="ttdef"><b>Definition:</b> <a href="_system_8cpp_source.html#l00278">System.cpp:278</a></div></div>
<div class="ttc" id="class_system_html_aba9d00b32578f2ea574bf67078db4aa7"><div class="ttname"><a href="class_system.html#aba9d00b32578f2ea574bf67078db4aa7">System::endMessage</a></div><div class="ttdeci">void endMessage(int cNum)</div><div class="ttdoc">プログラム終了時のメッセージを表示(c38) </div><div class="ttdef"><b>Definition:</b> <a href="_system_8cpp_source.html#l00060">System.cpp:60</a></div></div>
<div class="ttc" id="class_system_html_a5f235cfc9736103d150e9dbeee82e543"><div class="ttname"><a href="class_system.html#a5f235cfc9736103d150e9dbeee82e543">System::getProcessTimeinMiliseconds</a></div><div class="ttdeci">double getProcessTimeinMiliseconds()</div><div class="ttdoc">計測した時間をミリ秒単位で取得.startTimer()とendTimer()が実行されていることが前提(c65) </div><div class="ttdef"><b>Definition:</b> <a href="_system_8cpp_source.html#l00186">System.cpp:186</a></div></div>
<div class="ttc" id="_kinect_8hpp_html"><div class="ttname"><a href="_kinect_8hpp.html">Kinect.hpp</a></div></div>
<div class="ttc" id="_image_processing_8hpp_html"><div class="ttname"><a href="_image_processing_8hpp.html">ImageProcessing.hpp</a></div></div>
<div class="ttc" id="class_e_v3_control_html_a8ef402279dc736292ae0cf780dd3aed8"><div class="ttname"><a href="class_e_v3_control.html#a8ef402279dc736292ae0cf780dd3aed8">EV3Control::set6DoFEV3</a></div><div class="ttdeci">void set6DoFEV3(pcl::PointCloud&lt; pcl::PointXYZRGB &gt;::Ptr &amp;inputPointCloud, Point3d centroid, AttitudeAngle attitude_angle)</div><div class="ttdoc">最小二乗法によって求めた平均座標と位置をEV3の制御のために構造体に格納する(c80) </div><div class="ttdef"><b>Definition:</b> <a href="_e_v3_control_8cpp_source.html#l00049">EV3Control.cpp:49</a></div></div>
<div class="ttc" id="main_8cpp_html_a903d0d8820c696aaa1170c50deb3633f"><div class="ttname"><a href="main_8cpp.html#a903d0d8820c696aaa1170c50deb3633f">origin</a></div><div class="ttdeci">Point origin</div><div class="ttdoc">オリジナルの座標 </div><div class="ttdef"><b>Definition:</b> <a href="main_8cpp_source.html#l00027">main.cpp:27</a></div></div>
<div class="ttc" id="class_least_square_method_html_a19559c76884063927045732a38fe043b"><div class="ttname"><a href="class_least_square_method.html#a19559c76884063927045732a38fe043b">LeastSquareMethod::getCoefficient</a></div><div class="ttdeci">Eigen::Vector3f getCoefficient(pcl::PointCloud&lt; pcl::PointXYZRGB &gt;::Ptr &amp;inputPointCloud)</div><div class="ttdoc">最小二乗法によって平面ax+by+c=0の係数[a b c]&#39;を求めるメソッド </div><div class="ttdef"><b>Definition:</b> <a href="_least_square_method_8cpp_source.html#l00033">LeastSquareMethod.cpp:33</a></div></div>
<div class="ttc" id="class_system_html_a2f2eb72dffbb596c530ca149fd8830cd"><div class="ttname"><a href="class_system.html#a2f2eb72dffbb596c530ca149fd8830cd">System::checkDirectory</a></div><div class="ttdeci">void checkDirectory(const char *check_dirname)</div><div class="ttdoc">引数に与えたファイルやディレクトリが存在するかチェックし,無ければ作成する(c81) </div><div class="ttdef"><b>Definition:</b> <a href="_system_8cpp_source.html#l00233">System.cpp:233</a></div></div>
<div class="ttc" id="main_8cpp_html_a88f176e4e65e22a17093a8cd24003a66"><div class="ttname"><a href="main_8cpp.html#a88f176e4e65e22a17093a8cd24003a66">trackObject</a></div><div class="ttdeci">int trackObject</div><div class="ttdoc">追跡するオブジェクト </div><div class="ttdef"><b>Definition:</b> <a href="main_8cpp_source.html#l00026">main.cpp:26</a></div></div>
<div class="ttc" id="class_point_cloud_library_html_a2899ca3f4f6b2f8bce39524fba686e74"><div class="ttname"><a href="class_point_cloud_library.html#a2899ca3f4f6b2f8bce39524fba686e74">PointCloudLibrary::smoothingUsingMovingLeastSquare</a></div><div class="ttdeci">pcl::PointCloud&lt; pcl::PointXYZRGB &gt;::Ptr smoothingUsingMovingLeastSquare(pcl::PointCloud&lt; pcl::PointXYZRGB &gt;::Ptr &amp;inputPointCloud, bool compute_normals, bool polynomial_fit, double radius)</div><div class="ttdoc">スムージングを行うメソッド </div><div class="ttdef"><b>Definition:</b> <a href="_point_cloud_library_8cpp_source.html#l00198">PointCloudLibrary.cpp:198</a></div></div>
<div class="ttc" id="main_8cpp_html_a15633c47538e3790df1e008a7f6dbfea"><div class="ttname"><a href="main_8cpp.html#a15633c47538e3790df1e008a7f6dbfea">selection</a></div><div class="ttdeci">Rect selection</div><div class="ttdoc">選択 </div><div class="ttdef"><b>Definition:</b> <a href="main_8cpp_source.html#l00028">main.cpp:28</a></div></div>
<div class="ttc" id="_path_tracking_and_induction_of_the_robot_8hpp_html"><div class="ttname"><a href="_path_tracking_and_induction_of_the_robot_8hpp.html">PathTrackingAndInductionOfTheRobot.hpp</a></div></div>
<div class="ttc" id="class_system_html"><div class="ttname"><a href="class_system.html">System</a></div><div class="ttdoc">システム関連の処理を行うクラス </div><div class="ttdef"><b>Definition:</b> <a href="_system_8hpp_source.html#l00019">System.hpp:19</a></div></div>
<div class="ttc" id="class_least_square_method_html_adf062bbbd7b380abc5f24d3a09d39e47"><div class="ttname"><a href="class_least_square_method.html#adf062bbbd7b380abc5f24d3a09d39e47">LeastSquareMethod::calcYawRollPitch</a></div><div class="ttdeci">AttitudeAngle3d calcYawRollPitch(Eigen::Vector3f coefficient_plane)</div><div class="ttdoc">最小二乗法によって求めた[a b c]&#39;を用いて平面の姿勢を計算する </div><div class="ttdef"><b>Definition:</b> <a href="_least_square_method_8cpp_source.html#l00077">LeastSquareMethod.cpp:77</a></div></div>
<div class="ttc" id="class_point_cloud_library_html_a76c829d2feea3efcb84449a0c360ec43"><div class="ttname"><a href="class_point_cloud_library.html#a76c829d2feea3efcb84449a0c360ec43">PointCloudLibrary::extractplane_flag</a></div><div class="ttdeci">bool extractplane_flag</div><div class="ttdoc">平面検出とクラスタリングを行うかどうかのフラグ </div><div class="ttdef"><b>Definition:</b> <a href="_point_cloud_library_8hpp_source.html#l00058">PointCloudLibrary.hpp:58</a></div></div>
<div class="ttc" id="class_drawing_html_aa1da1f33436ebf7bfc05cf7453a6395e"><div class="ttname"><a href="class_drawing.html#aa1da1f33436ebf7bfc05cf7453a6395e">Drawing::gnuplotScriptTime2V</a></div><div class="ttdeci">void gnuplotScriptTime2V()</div><div class="ttdoc">時間と速度のプロット </div><div class="ttdef"><b>Definition:</b> <a href="_drawing_8cpp_source.html#l00093">Drawing.cpp:93</a></div></div>
<div class="ttc" id="class_point_cloud_library_html_a1ee163b625dc199c44149db55c84bafd"><div class="ttname"><a href="class_point_cloud_library.html#a1ee163b625dc199c44149db55c84bafd">PointCloudLibrary::flagChecker</a></div><div class="ttdeci">void flagChecker()</div><div class="ttdoc">フラグを判定するメソッド(c64) </div><div class="ttdef"><b>Definition:</b> <a href="_point_cloud_library_8cpp_source.html#l00077">PointCloudLibrary.cpp:77</a></div></div>
<div class="ttc" id="class_system_html_afce9f814824cbcb337abe755883f764c"><div class="ttname"><a href="class_system.html#afce9f814824cbcb337abe755883f764c">System::countdownTimer</a></div><div class="ttdeci">void countdownTimer(int time_sec)</div><div class="ttdoc">引数の時間[sec]に応じてカウントダウンを開始する(c75) </div><div class="ttdef"><b>Definition:</b> <a href="_system_8cpp_source.html#l00108">System.cpp:108</a></div></div>
<div class="ttc" id="class_point_cloud_library_html_a25f7509ec80771a394266cc98c117a9b"><div class="ttname"><a href="class_point_cloud_library.html#a25f7509ec80771a394266cc98c117a9b">PointCloudLibrary::statisticaloutlierremoval_flag</a></div><div class="ttdeci">bool statisticaloutlierremoval_flag</div><div class="ttdoc">外れ値を除去するかどうかのフラグ </div><div class="ttdef"><b>Definition:</b> <a href="_point_cloud_library_8hpp_source.html#l00056">PointCloudLibrary.hpp:56</a></div></div>
<div class="ttc" id="class_least_square_method_html_a3b3f2a5f264c576367c3fb3e5d70dcf2"><div class="ttname"><a href="class_least_square_method.html#a3b3f2a5f264c576367c3fb3e5d70dcf2">LeastSquareMethod::coefficient_plane</a></div><div class="ttdeci">Eigen::Vector3f coefficient_plane</div><div class="ttdoc">平面の係数 </div><div class="ttdef"><b>Definition:</b> <a href="_least_square_method_8hpp_source.html#l00029">LeastSquareMethod.hpp:29</a></div></div>
<div class="ttc" id="class_drawing_html_a98e39cb40a24d5909954c17fc1eaaa4f"><div class="ttname"><a href="class_drawing.html#a98e39cb40a24d5909954c17fc1eaaa4f">Drawing::gnuplotScriptEV3</a></div><div class="ttdeci">void gnuplotScriptEV3(int save_count, char *original_dirpath, char *output_filename, Eigen::Vector3f coefficient_plane)</div><div class="ttdoc">EV3の点群をプロットするためのスクリプト(c78) </div><div class="ttdef"><b>Definition:</b> <a href="_drawing_8cpp_source.html#l00049">Drawing.cpp:49</a></div></div>
<div class="ttc" id="class_system_html_af0fc7b12dc7342e2109a9aa3894b9bbe"><div class="ttname"><a href="class_system.html#af0fc7b12dc7342e2109a9aa3894b9bbe">System::showHelpMessage</a></div><div class="ttdeci">void showHelpMessage()</div><div class="ttdoc">キー入力に関するヘルプを表示(c86) </div><div class="ttdef"><b>Definition:</b> <a href="_system_8cpp_source.html#l00087">System.cpp:87</a></div></div>
<div class="ttc" id="class_point_cloud_library_html_aa8b11edec4c6910b0a9ad5c5340dcfa2"><div class="ttname"><a href="class_point_cloud_library.html#aa8b11edec4c6910b0a9ad5c5340dcfa2">PointCloudLibrary::centroid</a></div><div class="ttdeci">Point3d centroid</div><div class="ttdoc">平均座標 </div><div class="ttdef"><b>Definition:</b> <a href="_point_cloud_library_8hpp_source.html#l00042">PointCloudLibrary.hpp:42</a></div></div>
<div class="ttc" id="_point_cloud_library_8hpp_html"><div class="ttname"><a href="_point_cloud_library_8hpp.html">PointCloudLibrary.hpp</a></div></div>
<div class="ttc" id="class_e_v3_control_html"><div class="ttname"><a href="class_e_v3_control.html">EV3Control</a></div><div class="ttdoc">EV3を制御するためのクラス </div><div class="ttdef"><b>Definition:</b> <a href="_e_v3_control_8hpp_source.html#l00018">EV3Control.hpp:18</a></div></div>
<div class="ttc" id="class_kinect_html"><div class="ttname"><a href="class_kinect.html">Kinect</a></div><div class="ttdoc">Kinect操作用のクラス </div><div class="ttdef"><b>Definition:</b> <a href="_kinect_8hpp_source.html#l00030">Kinect.hpp:30</a></div></div>
<div class="ttc" id="class_least_square_method_html_a1ceee2394280e7d14411644de171782c"><div class="ttname"><a href="class_least_square_method.html#a1ceee2394280e7d14411644de171782c">LeastSquareMethod::attitude_angle</a></div><div class="ttdeci">AttitudeAngle3d attitude_angle</div><div class="ttdoc">姿勢角(c78) </div><div class="ttdef"><b>Definition:</b> <a href="_least_square_method_8hpp_source.html#l00031">LeastSquareMethod.hpp:31</a></div></div>
<div class="ttc" id="class_image_processing_html_a5fa5bf016a9d7f946919d0a34a47031c"><div class="ttname"><a href="class_image_processing.html#a5fa5bf016a9d7f946919d0a34a47031c">ImageProcessing::getUndistortionImage</a></div><div class="ttdeci">Mat getUndistortionImage(Mat &amp;inputOriginalImage)</div><div class="ttdoc">キャリブレーションデータを用いてKinectから取得した画像を補正する(c71) </div><div class="ttdef"><b>Definition:</b> <a href="_image_processing_8cpp_source.html#l00115">ImageProcessing.cpp:115</a></div></div>
<div class="ttc" id="class_point_cloud_library_html_a7c0232fc05e878dd58f64c1e6ed8e0b4"><div class="ttname"><a href="class_point_cloud_library.html#a7c0232fc05e878dd58f64c1e6ed8e0b4">PointCloudLibrary::initializePCLVisualizer</a></div><div class="ttdeci">void initializePCLVisualizer(string pclvisualizer_name)</div><div class="ttdoc">PCL Visualizerの初期化 </div><div class="ttdef"><b>Definition:</b> <a href="_point_cloud_library_8cpp_source.html#l00052">PointCloudLibrary.cpp:52</a></div></div>
<div class="ttc" id="main_8cpp_html_ab6ccc6c8d80970d4178c943533533860"><div class="ttname"><a href="main_8cpp.html#ab6ccc6c8d80970d4178c943533533860">onMouse</a></div><div class="ttdeci">void onMouse(int event, int x, int y, int flags, void *param)</div><div class="ttdoc">マウス操作 </div><div class="ttdef"><b>Definition:</b> <a href="_mouse_8cpp_source.html#l00019">Mouse.cpp:19</a></div></div>
<div class="ttc" id="stdafx_8h_html_a5d8ba032e2fdf7b2561f508164124f3e"><div class="ttname"><a href="stdafx_8h.html#a5d8ba032e2fdf7b2561f508164124f3e">NOC</a></div><div class="ttdeci">#define NOC</div><div class="ttdoc">Number of Characters.(ファイルの名前を付けるときの文字数制限) </div><div class="ttdef"><b>Definition:</b> <a href="stdafx_8h_source.html#l00081">stdafx.h:81</a></div></div>
<div class="ttc" id="class_image_processing_html_a58862a0cb31373a4d1a9d6a89fdd8e6c"><div class="ttname"><a href="class_image_processing.html#a58862a0cb31373a4d1a9d6a89fdd8e6c">ImageProcessing::loadInternalCameraParameter</a></div><div class="ttdeci">void loadInternalCameraParameter(const string cameraParamFile)</div><div class="ttdoc">カメラキャリブレーションによって得られたパラメータを適用する(c54) </div><div class="ttdef"><b>Definition:</b> <a href="_image_processing_8cpp_source.html#l00098">ImageProcessing.cpp:98</a></div></div>
<div class="ttc" id="class_e_v3_control_html_a206f5da82bb48c5ddae1906b9206d773"><div class="ttname"><a href="class_e_v3_control.html#a206f5da82bb48c5ddae1906b9206d773">EV3Control::getVelocityinSec</a></div><div class="ttdeci">void getVelocityinSec(double time_ms)</div><div class="ttdoc">EV3の速度を計算する(c85) </div><div class="ttdef"><b>Definition:</b> <a href="_e_v3_control_8cpp_source.html#l00067">EV3Control.cpp:67</a></div></div>
<div class="ttc" id="class_kinect_html_a975c1de77489cffc0728ae9817e127c2"><div class="ttname"><a href="class_kinect.html#a975c1de77489cffc0728ae9817e127c2">Kinect::streamEvent</a></div><div class="ttdeci">HANDLE streamEvent</div><div class="ttdoc">RGB,Depthカメラのフレーム更新イベントを待つためのイベントハンドル </div><div class="ttdef"><b>Definition:</b> <a href="_kinect_8hpp_source.html#l00052">Kinect.hpp:52</a></div></div>
<div class="ttc" id="class_point_cloud_library_html_abf4a5840ba484d07a775c2c0c9e41a99"><div class="ttname"><a href="class_point_cloud_library.html#abf4a5840ba484d07a775c2c0c9e41a99">PointCloudLibrary::downSamplingUsingVoxelGridFilter</a></div><div class="ttdeci">pcl::PointCloud&lt; pcl::PointXYZRGB &gt;::Ptr downSamplingUsingVoxelGridFilter(pcl::PointCloud&lt; pcl::PointXYZRGB &gt;::Ptr &amp;inputPointCloud, float leafSizeX, float leafSizeY, float leafSizeZ)</div><div class="ttdoc">ダウンサンプリングを行うメソッド </div><div class="ttdef"><b>Definition:</b> <a href="_point_cloud_library_8cpp_source.html#l00173">PointCloudLibrary.cpp:173</a></div></div>
<div class="ttc" id="class_image_processing_html_a36d146288e958736fd803fde49c064af"><div class="ttname"><a href="class_image_processing.html#a36d146288e958736fd803fde49c064af">ImageProcessing::openCVSettingTrackbar</a></div><div class="ttdeci">void openCVSettingTrackbar(const string trackbar_name)</div><div class="ttdoc">画像処理関連のトラックバーを表示するメソッド </div><div class="ttdef"><b>Definition:</b> <a href="_image_processing_8cpp_source.html#l00156">ImageProcessing.cpp:156</a></div></div>
<div class="ttc" id="class_point_cloud_library_html_a222d6644e01b9925d923d58eb7ea1522"><div class="ttname"><a href="class_point_cloud_library.html#a222d6644e01b9925d923d58eb7ea1522">PointCloudLibrary::getExtractPlaneAndClustering</a></div><div class="ttdeci">pcl::PointCloud&lt; pcl::PointXYZRGB &gt;::Ptr getExtractPlaneAndClustering(pcl::PointCloud&lt; pcl::PointXYZRGB &gt;::Ptr &amp;inputPointCloud, bool optimize, int maxIterations, bool negative1, bool negative2, int minClusterSize, int maxClusterSize)</div><div class="ttdoc">平面検出とクラスタリング </div><div class="ttdef"><b>Definition:</b> <a href="_point_cloud_library_8cpp_source.html#l00229">PointCloudLibrary.cpp:229</a></div></div>
</div><!-- fragment --></div><!-- contents -->
</div><!-- doc-content -->
<!-- start footer part -->
<div id="nav-path" class="navpath"><!-- id is needed for treeview function! -->
  <ul>
    <li class="navelem"><a class="el" href="dir_45aeffd6702719c5f3cbf0d005461816.html">SkyDrive</a></li><li class="navelem"><a class="el" href="dir_a0d9d77de56eed61b71feb2adefbb47d.html">ShigeLab</a></li><li class="navelem"><a class="el" href="dir_89d32e4e26b9b93c85c915e3b07e88b7.html">Programming</a></li><li class="navelem"><a class="el" href="dir_f222f4b39d8b1349931d836df5dbea5a.html">Master</a></li><li class="navelem"><a class="el" href="dir_3890cccb326c0ee1b484a4cf99bcc311.html">PathTrackingAndInductionOfTheRobot</a></li><li class="navelem"><a class="el" href="dir_d2cbe6861c52eb610edd18736eb43ce0.html">PathTrackingAndInductionOfTheRobot</a></li><li class="navelem"><a class="el" href="main_8cpp.html">main.cpp</a></li>
    <li class="footer">2016年01月25日(月) 12時00分32秒作成 - PathTrackingAndInductionOfTheRobot / 構成: 
    <a href="http://www.doxygen.org/index.html">
    <img class="footer" src="doxygen.png" alt="doxygen"/></a> 1.8.6 </li>
  </ul>
</div>
</body>
</html>
